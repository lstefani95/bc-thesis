\documentclass[
  digital, %% This option enables the default options for the
           %% digital version of a document. Replace with `printed`
           %% to enable the default options for the printed version
           %% of a document.
  twoside, %% This option enables double-sided typesetting. Use at
           %% least 120 g/m² paper to prevent show-through. Replace
           %% with `oneside` to use one-sided typesetting; use only
           %% if you don’t have access to a double-sided printer,
           %% or if one-sided typesetting is a formal requirement
           %% at your faculty.
  notable,   %% This option causes the coloring of tables. Replace
           %% with `notable` to restore plain LaTeX tables.
  lof,     %% This option prints the List of Figures. Replace with
           %% `nolof` to hide the List of Figures.
  lot,     %% This option prints the List of Tables. Replace with
           %% `nolot` to hide the List of Tables.
  %% More options are listed in the user guide at
  %% <http://mirrors.ctan.org/macros/latex/contrib/fithesis/guide/mu/econ.pdf>.
]{fithesis3}
%% The following section sets up the locales used in the thesis.
\usepackage[resetfonts]{cmap} %% We need to load the T2A font encoding
\usepackage[utf8]{inputenc}
\usepackage[T1]{fontenc}  %% to use the Cyrillic fonts with Russian texts.
\usepackage[
  main=slovak, %% By using `czech` or `slovak` as the main locale
                %% instead of `english`, you can typeset the thesis
                %% in either Czech or Slovak, respectively.
  english, german, russian, slovak %% The additional keys allow
]{babel}        %% foreign texts to be typeset as follows:
%%
%%   \begin{otherlanguage}{german}  ... \end{otherlanguage}
%%   \begin{otherlanguage}{russian} ... \end{otherlanguage}
%%   \begin{otherlanguage}{czech}   ... \end{otherlanguage}
%%   \begin{otherlanguage}{slovak}  ... \end{otherlanguage}
%%
%% For non-Latin scripts, it may be necessary to load additional
%% fonts:
\usepackage{paratype}



%%
%% The following section sets up the metadata of the thesis.
\thesissetup{
    date               = \the\year/\the\month/\the\day,
    autoLayout         = false,
    university         = mu,
    faculty            = econ,
    type               = bc,
    field              = Hospodárska politika,
    department         = Katedra ekonómie,
    author             = Lenka Štefanidesová,
    gender             = f,
    advisor            = {Ing. Rostislav Staněk, Phd.},
    extra              = {
      advisorSkGenitiv = {Ing. Rostislava Staňeka, Phd.}
    },
    title              = { \makecell[l]{Motivačný efekt priebežného výsledku: \\ Robustnosť evidencie zo športových zápasov}},
    TeXtitle           = Motivačný efekt priebežného výsledku: Robustnosť evidencie zo športových zápasov,
    keywords           = {basketbal, Berger a Pope, motivačný efekt, model regresnej diskontinuity},
    TeXkeywords        = {basketbal, Berger a Pope, motivačný efekt, model regresnej diskontinuity},
    abstract           = {Bakalárska práca "Motivačný efekt priebežného výsledku: Robustnosť evidencie zo športových zápasov" sa zameriava na motivačný efekt prehry v priebehu zápasu. Na dátach z piatich basketbalových líg je overovaná robustnosť výsledkov štúdie Bergera a Popa (2011), ktorá tvrdí, že prehrávanie v polčase môže viesť k výhre. V súvislosti s motivačným efektom v práci rozoberáme rozdielnosť medzi mužským a ženským basketbalom a pričom taktiež zohľadňujeme rolu favorita.},
    thanks             = {Týmto by som sa chcela poďakovať Ing. Rostislavovi Staňekovi, Phd. za vedenie práce, pomoc pri jej vypracovávaní a cenné postrehy, pripomienky a rady. 
    	
    Takisto ďakujem Michalovi za jeho nekonečnú trpezlivosť, ochotu a podporu, bez ktorej by táto práca vznikla len veľmi ťažko.},
    bib                = bibliografia.bib,
    %% Uncomment the following line (by removing the % symbol at
    %% the beginning) and replace `assignment.pdf` with the
    %% filename of your scanned thesis assignment.
    assignment    = zadanie.pdf,
    %% The following keys are only useful, when you're using a
    %% locale other than English. You can safely omit them in an
    %% English thesis.
    titleEn            = { \makecell[l]{Incentive effect of intermediate result: \\ Robustness check of sport matches evidence}},
    TeXtitleEn         = Incentive effect of intermediate result: Robustness check of sport matches evidence,
    keywordsEn         = {basketball, Berger and Pope, incentive effect, regression discontinuity design},
    TeXkeywordsEn      = {basketball, Berger and Pope, incentive effect, regression discontinuity design},
    abstractEn         = {Bachelor thesis "Incentive effect of intermediate result: Robustness check of sport matches evidence" focuses on the incentive effect of the loss during the match. Using dataset coming from five basketball leagues we are investigating the robustness of the results of the study by Berger and Pop (2011), who claim that loosing in half-time break may lead to winning. In the context of the incentive effect, we discuss the difference between male and female basketball, and then we take into account the role of the favorite.},
}
\usepackage{makeidx}      %% The `makeidx` package contains
\makeindex                %% helper commands for index typesetting.

%% These additional packages are used within the document:
\usepackage{paralist} %% Compact list environments
\usepackage{amsmath}  %% Mathematics
\usepackage{amsthm}
\usepackage{amsfonts}
\usepackage{url}      %% Hyperlinks
\usepackage{markdown} %% Lightweight markup
\usepackage{listings} %% Source code highlighting
\lstset{
  basicstyle      = \ttfamily,%
  identifierstyle = \color{black},%
  keywordstyle    = \color{blue},%
  keywordstyle    = {[2]\color{cyan}},%
  keywordstyle    = {[3]\color{olive}},%
  stringstyle     = \color{teal},%
  commentstyle    = \itshape\color{magenta}}
\usepackage{floatrow} %% Putting captions above tables
\floatsetup[table]{capposition=top}
\floatsetup[figure]{capposition=top}
\usepackage{chngcntr}
\counterwithout{table}{chapter}  % Flat numbering of tables.
\counterwithout{figure}{chapter} % Flat numbering of figures.

\usepackage{makecell}
\usepackage{multirow}
\usepackage{hyperref}
\usepackage[noabbrev,capitalise]{cleveref}
\usepackage{textcomp}
\usepackage[figureposition=top]{caption}
\usepackage{setspace}
	\setstretch{1.5} %% Nastavenie riadkovania.
\usepackage{xcolor}


\begin{document}
	\renewcommand{\listfigurename}{Zoznam grafov}
	\makeatletter
	\thesis@preamble %% Print the preamble.
	\makeatother
	
	\chapter*{Úvod}
	\addcontentsline{toc}{chapter}{Úvod}
	
	Ekonómiže byť testovanie v prírodných vedách. Nie je to však nemožné, využitím či už historických dát, experimentálnych pokusov aleboa je spoločenská veda, v ktorej nie je testovanie a overovanie hypotéz a predpokladov také jednoduché, akým mô  
	
	Bakalárská práca  s názvom \textit{Motivačný efekt priebežného výsledku: Robustnosť evidencie zo športových zápasov} sa zameriava na skúmanie športových dát, ktoré predstavujú vhodné laboratórium pre štúdium motivačných efektov.
	
	Práca vychádza primárne zo štúdie Bergera a Popa (2011), ktorí na základe dát z basketbalových zápasov ukazujú, že tesná prehra v priebehu hry, špecificky v polčase, má silný motivačný efekt. 
	
	Cieľom tejto práce je na dátach z viac ako dvadsaťjedentisíc  basketbalových zápasov a piatich rôznych líg overiť robustnosť tohto ich záveru. Okrem toho je cieľom preskúmať ako sa líši efekt priebežného výsledku medzi pohlaviami, tj. či sa vyskytuje v mužských a aj ženských súťažiach, prípadne na ktoré pohlavie vplýva silnejšie. Očakávanie je, že efekt by mal byť zhruba rovnaký pre obe pohlavia. Posledným cieľom tejto práce je porovnať vplyv motivačného efektu pre skupinu favoritov a skupinu tých druhých, nie tak silne favorizovaných tímov. Motiváciou tohto skúmania je overenie predpokladu, že motivačný efekt bude výraznejší pre favoritov.
	
	Z toho vychádzajúc, táto bakalárska práca hľadá odpovede na otázky: \textit{Sú výsledky štúdie Berger a Popa robustné? Funguje motivačný efekt rovnako pre mužov ako aj pre ženy? Je rozdiel vo výskyte motivačného efekte medzi favoritmi a tými ostatnými?} 
	
	Práca začína krátkym úvodom do problematiky prepojenia športu a ekonómie a následne v stručnosti uvádza čitateľa do problematiky  motivačného efektu. Literárnu rešerš teórie motivačného efektu tvoria okrem štúdie Bergera a Popa ďalších päť prác venujúcim sa obdobnej tematike. Rešerš sa zameriava na vecné zhrnutie stanovených cieľov, zvolených metód riešenia, využitých nástrojov a dosiahnutých výsledkov. 
	
	Druhá kapitola je krátkym úvodom do basketbalu ako športu, v rámci ktorej sú v stručnosti popísané základné pravidlá a charakteristika piatich basketbalových líg, ktorých dáta sú v práci použité.
	
	Pre účely zistenia odpovedí na otázky tejto práce bolo nutné spracovať dáta metódou regresnej diskontinuity. Pôvodný dataset pozorovaní bol podrobený úprave a prefiltrovaniu aby vyhovoval potrebám zostrojeného ekonometrického modelu, jeho vysvetľovanej a vysvetľujúcim premenným.
	
	Posledná kapitola obsahuje spracovanie dát a popis získaných výsledkov. V závere sú zhrnuté získané poznatky a zodpovedané nastolené otázky z úvodu práce.
	

	\chapter{Prepojenie športu a ekonómie}
	Táto kapitola predstavuje rôzne prepojenia dvoch na prvý pohľad nijako nesúvisiacich oblastí, akými sú šport a ekonómia.
	
	Šport ako taký, či už na profesionálnej alebo amatérskej úrovni, v súčasnosti zaujal pomerne prominentné miesto v spoločnosti. V podobe zábavného priemyslu má celosvetový dosah na milióny ľudí, ktorí sa nielen priamo zúčastňujú športových podujatí ale aj čítajú športové články a časopisy, či si predplácajú športové televízne stanice a pravidelne sledujú dianie svojho obľúbeného športu. \parencite{conrad2011} Okrem toho sú ľudia denno-denne vyzývaní aby sa sami venovali nejakej športovej aktivite, ktorá je vyzdvihovaná pre svoje zdravotné benefity.
	
	Športová ekonómia je podľa Bootha (\citeyear[s.~377]{booth2009}) kombináciou niekoľkých oblastí ekonómie, počínajúc napríklad ekonómiou práce, verejnými financiami alebo mikroekonomickými princípmi použitými v súvislosti so športovou organizáciou a priemyslom. Okrem toho sa športová ekonómia môže zaoberať aj témami rasovej a rodovej diskriminácie alebo reálnym ekonomickým dopadom konania športových udalostí či stavby športových zariadení. 
	
	Mimoriadne významná je skutočnosť, že pre ekonómiu ako vedu je vo všeobecnosti pomerne obtiažne a často krát dokonca nemožné  priamo testovať svoje teórie a hypotézy. Šport sa však vyznačuje jasne zadefinovanými pravidlami a generovaním veľkého množstva dát. Do určitej miery sa dá šport považovať za ideálne prostredie pre niektoré ekonomické výskumy a testy, je akýmsi laboratóriom.
	
	Práce, ktoré prepájajú ekonómiu a šport sa môžu zameriavať napríklad na profesionálne tímové športy ako basketbal, hokej alebo americký futbal, keďže tie sú typické práve už spomenutým veľkým objemom najrôznejších kvantitatívnych dát, ktoré pozostávajú nielen z tímových charakteristík, ale aj charakteristík jednotlivcov v rámci súťaže. Ide napríklad o bodovanie, držanie lopty alebo počet odohraných minút v hre.
	
	Autorov zaoberajúcich sa športovou ekonómiou je mnoho a ich práce sa líšia nielen poňatím spracovaných tém, ale napríklad aj náročnosťou textu. Publikácia od Leedsa (\citeyear{leeds2018}) obsahuje niekoľko konkrétnych príkladov aplikovania mikroekonómie a jej konceptov a teórií priamo na americké a medzinárodné športy. Késenne (\citeyear{kesenne2014}) sa vyznačuje analytickým poňatím teórie profesionálnych tímových športov a autori Sandy, Sloane a Rosentraub (\citeyear{sandy2004}) sa zameriavajú najmä na prípadové štúdie v súvislosti so Spojenými štátmi americkými, Kanadou, Európou a Austráliou. Odlišnou je publikácia Kupera a Syzmanskeho (\citeyear{kuper2009}), ktorá cez ekonómiu, štatistiku, psychológiu a teóriu hier odpovedá na mnohé otázky týkajúceho sa futbalu, napr. prečo vo futbale krajiny ako Brazília, Nemecko a Španielsko vyhrávajú ale Anglicko nie. Foer (\citeyear{foer2004}) vo svojej publikácii používa futbal ako prostriedok na vysvetlenie tém akými sú napríklad globalizácia, antisemitizmus či nacionalizmus. Mnoho autorov sa v súvislosti so športom zaoberá motiváciou a teóriou motivačného efektu, ktorá je predmetom aj tejto bakalárskej práce. Preto sa nasledujúca podkapitola venuje práve motivačnému efektu.
	
		\section{Teória motivačného efektu}
		Motivácia by sa dala vo všeobecnosti definovať ako akýsi vnútorný podnet jednotlivca k určitým aktivitám, správaniu a snahe dosahovať vytýčené ciele. 
		
		Čo sa týka motivácie k súťaženiu, existuje niekoľko dôvodov prečo súťažiť. Podľa Frankena a Browna (\citeyear[s.~176]{franken1995}) niektorých ľudí motivuje možnosť uspokojiť si, prostredníctvom súťaže a konkurencie v rámci nej, potrebu výhry. Týmto spôsobom chcú dokázať svoju nadradenosť nad ostatnými. \parencite{nicholls1989} Na druhú stranu sú však aj ľudia, ktorých motivuje možnosť zlepšenia svojich schopností, tzn. že chcú byť lepší ako svoje včerajšie ja.
		
		Motivácia a súťaživosť, či už v športe alebo v živote ako takom, sa spájajú takmer automaticky. Súťaženie na seba môže brať mnoho podôb, či už ide o získanie najlepšej známky v triede, víťazstvo prvého miesta v športovom turnaji alebo získanie finančnej odmeny za najlepší predaj. \parencite[s.~210]{tauer1999}
		
		Motivácia, ale aj jej opak demotivácia, sú veľmi silnými hybnými silami, ktoré vedú k ovplyvneniu psychickej pohody a fyzického výkonu. Táto práca sa venuje motivácií v súvislosti s basketbalovými zápasmi a tomu, aký je jej motivačný efekt na výsledok zápasu pri prehrávaní v polčase. Teória motivačného efektu v športe sa stala predmetom skúmania viacerých autorov. 
		
		Niektorí autori skúmali výrazné vyhrávanie tímov, napríklad Cooper (\citeyear{cooper1992}), podľa ktorého basketbalové tímy, ktoré sú výraznejšie popredu už v úvode hry vyhrávajú približne dve tretiny času. Ďalší autori sa zamerali na motivovanosť tímov, ktoré prehrávajú tesne ale aj výraznejšie. Nasledujúca podkapitola obsahuje dovedna šesť prác, ktoré tvoria prehľad literatúry a rôznych prístupov a využitých nástrojov a metód viažúcich sa k téme tejto bakalárskej práce, tzn. motivačnému efektu. 
		
		\section{Literárna rešerš}
		\subsection{Berger a Pope}
		\label{sec:Berger}
		Hlavnou prácou, o ktorú sa táto bakalárska práca opiera je článok Bergera a Popa (\citeyear{berger2011}) s názvom \textit{Can Loosing Lead to Winning?}, v preklade \textit{Môže prehrávanie viesť k víťazstvu?}, ktorý má jednoznačný cieľ a to ukázať na reálnych dátach, že byť pozadu môže v konečnom dôsledku viesť prostredníctvom motivácie k zvýšeniu šance na výhru.
		
		Rovnako ako táto bakalárska práca, aj Berger a Pope spracovávali dáta z basketbalových zápasov a skúmali pomocou modelu regresnej diskontinuity ako prehrávanie ovplyvňuje motiváciu a výkon. Regresná diskontinuita je bližšie vysvetlená v časti \ref{sec:rdd}.
		
		Dataset v štúdii tvorilo viac ako 18 000 zápasov profesionálnej NBA, v rozmedzí sezóny 1993/1994 a marca 2009, a viac ako 45 000 zápasov NCAA, Americkej univerzitnej basketbalovej ligy, ktoré sú z obdobia medzi sezónou 1999/2000 a marcom 2009. \parencite[s.~818]{berger2011} Dáta obsahujú okrem informácie kto bol víťazom zápasu aj údaje o bodovom skóre oboch tímov v polčase.
		
		Využitý ekonometrický model mal nasledovnú podobu:
		\begin{equation}
		\label{eq:bp}
		\textit{výhra}_{i} = \alpha + \beta [\textit{prehrávanie~v~polčase}]_{i} + \delta [\textit{bodový~rozdiel~v~polčase}]_{i} + \gamma X_{i} + \varepsilon_{i},
		\end{equation}
		
		kde \textit{výhra$_{i}$} predstavuje umelú vysvetľovanú premennú, ktorá nadobúda hodnotu jedna keď domáci tím zápas vyhral a nula keď ho prehral. Premenná \textit{prehrávanie~v~polčase$ _{i} $} je vysvetľujúca premenná, ktorá je taktiež umelou premennou a nadobúda hodnotu jedna keď domáci tím prehráva v polčase o jeden a viac bodov a nula keď vyhráva. Druhou vysvetľujúcou premennou je \textit{bodový~rozdiel~v~polčase$ _{i} $}, ktorá pozostáva z rozdielu v skóre medzi domácim a hosťujúcim tímom. \textit{Xi} v modeli reprezentuje maticu kontrolných premenných, ktorými sú sezónne výherné percentá domáceho a hosťujúceho tímu.  V modeli sa nachádza aj $\alpha$ ako úrovňová konštanta a $\varepsilon_{i}$, čiže náhodná zložka.
		
		Autori v práci menujú niekoľko dôvodov, prečo sa ich pozornosť zameriava práve na polčas. Vyzdvihujú najmä spätnú väzbu v polčase. Vo všeobecnosti spätná väzba pomáha ľuďom upraviť ich postup a snahu tak, aby boli schopní dosiahnuť stanovený cieľ. \parencite{locke2002} Hlavným cieľom v akomkoľvek športe, takže aj v basketbale, je výhra.
		
		Počas dlhšej pätnásť minútovej prestávky v polčase, oproti stotridsať sekundovej prestávke medzi prvou a druhou a treťou, resp. štvrtou štvrtinou, si jednotliví hráči lepšie uvedomia súčasný stav a z neho vyplývajúci bodový rozdiel. Taktiež si vedia lepšie predstaviť, aké veľké zlepšenie bude potrebné na zníženie súperovho náskoku. Okrem toho je to ideálna príležitosť trénerov zmeniť taktiku hry, prehovoriť hráčom tzv. do duše a vyburcovať ich k lepšiemu výkonu. Keďže je basketbal tímový šport, polčas je vhodným momentom pre utuženie tímového ducha, prediskutovanie prípadných nejasnosti týkajúcich sa spolupráce hráčov a zvýšenie celkovej motivácie.
		
		\renewcommand{\figurename}{Graf}
		\begin{figure}[h]
			\centering
			\includegraphics[width=0.8\textwidth]{./grafy/NBA.png}
			\caption{Percento vyhraných zápasov domáceho tímu na základe bodového rozdielu pre dáta z NBA – štúdia Berger a Pope}
			\vskip\abovecaptionskip\emph{Zdroj: <<\cite{berger2011} a vlastné spracovanie>>}
			\label{fig:NBA1}
			\vskip\abovecaptionskip\emph{Pozn.: Nespracované dáta predstavujú čierne body a prerušovaná čiara znázorňuje lineárny tvar dát.}
		\end{figure}
	
		Výsledky štúdie v oblasti zápasov NBA potvrdzujú predpoklad autorov z úvodu práce, že byť trochu pozadu môže viesť k výhre. Ako vidieť na grafe \ref{fig:NBA1}, ktorý je vykreslením vzťahu medzi percentom  zápasov vyhraných domácim tímom a rozdielom v skóre domáceho a hosťovského tímu, v bode nulového bodového rozdielu existuje diskontinuita. 
		
		Výsledky testovania vplyvu prehrávania v polčase na výhru v NBA dátach sú zobrazené v tabuľke \ref{table:vplyvNBA1}. Stĺpce s označením (1), (2), (3) a (4) predstavujú štyri rôzne variácie základného modelu \ref{eq:bp}. Modely v stĺpcoch (1) a (3) obsahujú premennú \textit{score~difference~at~halftime} iba v jej lineárnej podobe, zatiaľ čo modely (2) a (4) obsahujú aj jej kubickú podobu. Modely (1) a (2) neobsahujú žiadnu ďalšiu kontrolnú premennú, modely (3) a (4) však zahŕňajú aj premennú sezónnych výherných percent domáceho a hosťujúceho tímu. Spojovník v tabuľke reprezentuje skutočnosť, že daná premenná sa v modeli nevyskytuje, zatiaľ čo premenné so symbolom X sú síce v modeli zahrnuté, avšak autori nezverejnili ich hodnoty. 
		
		Na spomenutom grafe \ref{fig:NBA1} je vidieť, že čím viac tím vyhráva, resp. prehráva v polčase, tým je pravdepodobnejšie, že v konečnom dôsledku vyhrá, resp. prehrá. Avšak výsledky ukazujú aj to, že tímy, ktoré prehrávajú v polčase iba o trochu, o jeden až dva body, vo výsledku vyhrávajú častejšie o 5,8 až 8,0 percentných bodov, viď tabuľka \ref{table:vplyvNBA1}. 
		
		Berger a Pope hovoria, že namiesto toho, aby domáci tím prehrávajúci o bod mal o približne 5 až 8 percentných bodov nižšiu pravdepodobnosť výhry, v skutočnosti je pravdepodobnejším víťazom celého zápasu s pravdepodobnosťou 58,2 \% prípadoch.
		
		\newcolumntype{x}[1]{>{\centering\hspace{0pt}}p{#1}}
		\newcommand{\sizeofcolumn}{5.4em}
		\begin{table}[t]
			\catcode`\-=12
			\centering 
			\bgroup
			\def\arraystretch{1.5}
			\begin{tabular}{|l||x{\sizeofcolumn}x{\sizeofcolumn}x{\sizeofcolumn}x{\sizeofcolumn}|}
				\hline
				
				\multirow{2}{*}{} & \multicolumn{4}{c|}{Závislá premenná:  výhra$_{i} $ = 1 ak domáci tím vyhral zápas} \tabularnewline \cline{2-5}
				& (1) & (2) & (3) & (4) \tabularnewline \hline \hline
				
				\multirow{2}{*}{\textit{Prehrávanie v polčase}} & 0,058*** & 0,074*** & 0,062*** & 0,080*** \tabularnewline
				& (0,015) & (0,021) & (0,015) & (0,020) \tabularnewline
				\hline
				
				\textit{Domáci tím} &  &  & 0,0068*** & 0,0068*** \tabularnewline
				\multicolumn{1}{|r||}{\textit{výherné percento}} & & & (0,0002) & (0,002) \tabularnewline
				\hline 
				
				\textit{Hosťujúci tím} &  &  & -0,0065*** & -0,0065*** \tabularnewline
				\multicolumn{1}{|r||}{\textit{výherné percento}} & & & (0,0002) & (0,002) \tabularnewline
				\hline
				
				\textit{Bodový rozdiel v polčase} & X & X & X & X \tabularnewline
				\multicolumn{1}{|r||}{(\textit{lineárny})} & & & & \tabularnewline
				\hline
				
				\textit{Bodový rozdiel v polčase} & & X & & X \tabularnewline
				\multicolumn{1}{|r||}{(\textit{kubický})} & & & & \tabularnewline
				\hline
				
				$ Pseudo-R^{2} $ & 0,097 & 0,097 & 1,172 & 1,172 \tabularnewline
				\hline

				\textit{Pozorovania} & 11 968 & 11 968 & 11 968 & 11 968 \tabularnewline
				\hline
				
			\end{tabular}
			\egroup
			\caption{Vplyv prehrávania v polčase na výhru v NBA dátach}
			\vskip\abovecaptionskip\emph{Zdroj: <<\cite{berger2011} a vlastné spracovanie>>}
			\vskip\abovecaptionskip\emph{Pozn.: V tabuľke sú uvedené hodnoty medzného efektu koeficientov jednotlivých premenných a v zátvorkách je ich smerodatná chyba. Použitý logit model.}
			\label{table:vplyvNBA1}
		\end{table}	
	
		Okrem NBA ligy skúmali autori aj univerzitnú ligu NCAA a pokúšali sa zistiť, či sa motivačný efekt bytia pozadu vyskytuje aj v nej. Použitý bol rovnaký typ dát ako aj rovnaké metódy skúmania. Výsledkom je potvrdenie výskytu motivačného efektu aj v tejto vzorke pozorovaní, aj keď vplyv motivačného efektu v tomto prípade nie je až taký výrazný. Toto je znázornené nielen na grafe \ref{fig:NCAA1}, kde je diskontinuita zobrazená menším skokom v bode nulového bodového rozdielu, ale aj v tabuľke \ref{table:vplyvNCAA}, v nižších medzných hodnotách koeficientov premennej prehrávania v polčase.
		
		Po overení a potvrdení predpokladu o fungovaní motivačného efektu bytia pozadu sa autori rozhodli zistiť aj to, kedy má tento efekt najväčší vplyv. Očakávanie autorov bolo, že najväčší vplyv efektu je hneď po polčase, tj. v tretej tretine.
		
		Logit modely vytvorené na overenie tohto predpokladu sa svojou štruktúrou zhodovali s už skôr použitými modelmi s tým rozdielom, že vysvetľovanou premennou nebola výhra celého zápasu ale výhra v tretej, resp. vo štvrtej štvrtine. To znamená, že vysvetľované premenné indikovali skutočnosť, či mal domáci tím viac bodov na konci danej štvrtiny ako súper. 
		
		Znamienka koeficientov pre premennú prehrávania v polčase sú v oboch prípadoch, pre tretiu aj štvrtú štvrtinu, kladné. Prehrávanie v polčase tým pádom zvýšilo pravdepodobnosť výhry v jednotlivých štvrtinách, avšak väčšie a štatisticky významné sú iba koeficienty pre tretiu tretinu, čím sa predpoklad autorov potvrdil. \parencite[s.~822]{berger2011}		
			
		Za zmienku stojí upozornenie autorov, že motivačný efekt polčasového prehrávania je približne polovičnej významnosti ako všeobecne populárny efekt domácej výhody.

		\subsection{Merritt a Clauset}
		Ďalším športom, ktorý bol podrobený skúmaniu v súvislosti s teóriou motivačného efektu. Štúdia Merritta a Clauseta (\citeyear{merritt2014}) sa okrem basketbalu venovala aj americkému futbalu a hokeju. Autori sa zaoberajú pravdepodobnosťou ďalšieho skórovania pri vyhrávaní, resp. prehrávaní a skúmajú, či veľkosť vedenia v zápase poskytuje informáciu o tom, kto získa ďalšie body, tzn. dynamiku skórovania. To samo o sebe nie je predmetom tejto bakalárskej práce, avšak výsledky štúdie \textit{Scoring dynamics across professional team sports: tempo, balance and predictability} obsahujú zaujímavé poznatky aj z oblasti motivačného efektu prepojeného na basketbal.
		
		Dataset tejto štúdie je pomerne rozsiahly a predstavujú ho zápasy počas 9 až 10  sezón štyroch amerických športových líg: CFL\footnote{Skratka autorov, ide o Univerzitnú ligu amerického futbalu.}, NFL\footnote{Celý názov National Football League, v preklade Národná futbalová liga.}, NBA a NHL\footnote{Celý názov National Hockey League, v preklade Národná hokejová liga.}. V štúdii autori pracujú s viac ako 40 000 zápasmi, ktorých prehľad obsahuje tabuľka \ref{table:summary1;}. Tieto zápasy sú spracované kombináciou Bernoulliho a Poissonového procesu.
	
		\begin{table}[h]
			\catcode`\-=12
			\centering	
			\bgroup
			\renewcommand{\arraystretch}{1} 
			\def\arraystretch{1.3}
			\begin{tabular}{|l|c|c|c|c|c|}
				\hline
				\textbf{Šport} & \textbf{Typ} & \textbf{Skratka} & \textbf{Sezóna} & \makecell{\textbf{Počet} \\ \textbf{zápasov}} & \makecell{\textbf{Počet} \\ \textbf{skórovaní}} \\ \hline \hline
				Americký futbal & univerzitný & CFB & 10, 2000-2009 & 14 588 & 120 827 \\
				Americký futbal & pro & NFL & 10, 2000-2009 & 2 654 & 19 476 \\
				Hokej & pro & NHL & 10, 2000-2009 & 11 813 & 44 989 \\
				Basketbal & pro & NBA & 9, 2002-2010 & 11 744 & 1 080 285 \\ \hline				
			\end{tabular}
			\egroup
			\caption{Prehľad dát pre každý šport}
			\vskip\abovecaptionskip\emph{Zdroj: <<\cite{merritt2014} a vlastné spracovanie>>}
			\label{table:summary1;}
		\end{table}
	
		Výsledky štúdie nie sú totožné naprieč všetkými štyrmi ligami. Pravdepodobnosť ďalšieho skórovania sa v CFL, NFL a NHL zvyšuje s výškou vedenia, zatiaľ čo v prípade NBA, teda basketbalu, sa táto pravdepodobnosť znižuje. Plynutím hry tak môže dôjsť k zmenšovaniu vedúceho náskoku až kým sa vedenie nezmení v remízu, čo vytvára z basketbalu akúsi nepredvídateľnú hru. Podľa autorov je jedným z možných vysvetlení práve motivačný efekt, kedy sa prehrávajúci tím snaží viac, tzn. že pravdepodobnosť skórovania je vyššia. Nie je však jasné, prečo sa tento fenomén objavil iba v prípade NBA. \parencite[s.~18]{merritt2014}
	
		Autori však ponúkajú zdôvodnenie, podľa ktorého to môžu spôsobovať stratégie tímov NBA, ktoré sú špecifické pre tento šport. Ako príklad prezentovali stiahnutie lepších a skúsenejších hráčov zo zápasu pri vedení, aby sa zbytočne neprepínali  alebo aby sa nezranili. Bez týchto hráčov sa pravdepodobnosť skórovania pre vedúci tým znižuje. Oproti tomu napríklad v hokeji hráči rotujú v jednotlivých formáciách a v americkom futbale takéto „náhrady“ ako v basketbale nie sú časté. \parencite[s.~19]{merritt2014}
	
		\subsection{Bergerhoff a Vosen}
		Bergerhoff a Vosen (\citeyear{bergerhoff2015}) sa taktiež zameriavajú na fenomén, kedy byť pozadu môže byť za určitých okolností výhodnejšie. Špecifikom tejto štúdie je testovanie predmetu skúmania prostredníctvom modelovej situácie, tzn. bez použitia reálnych dát.
		
		Zistenia autorov sú, že hráčov motivujú k prekonaniu počiatočnej nevýhodu dve skutočnosti. Buď cena turnaja, ligy alebo súťaže v podobe finančného ohodnotenie, trofeje či prestíže z výhry alebo tesná hra sama o sebe. Hráči následne vynakladajú viac úsilia a preto majú vyššiu pravdepodobnosť výhry. \parencite[s.~20]{bergerhoff2015} Dalo by sa povedať, že keď v začiatku určité znevýhodnenie jednej strany spôsobí silné zvýhodnenie druhú stranu, tá sa stáva motivovanejšou a vyvíja väčšiu snahu. Avšak v prípade, keď je rozdiel v skóre minimálny a zápas je tesný, ten tím, ktorý momentálne mierne stráca sa ukazuje ako motivovanejší a snaživejší.
		
		Tieto výsledky, ako uznávajú aj samotní autori, vysvetľujú a podporujú závery Bergera a Popa (\citeyear{berger2011}), ktoré sú spomenuté vyššie v oddieli \ref{sec:Berger}.
		
		\subsection{Schneemann}
		Štúdia, ktorá sa výsledkom nezhoduje s prácou Bergera a Popa (\citeyear{berger2011}) je štúdia Schneemanna a Deutschera (\citeyear{schneemann2017}). Dataset v tomto prípade tvorili zápasy nemeckej futbalovej Bundesligy a to z dvoch sezón  2011/2012 a 2013/2014, čo dovedna tvorí 918 zápasov. Autori na spracovanie dát využívali deskriptívnu štatistiku a regresnú analýzu.
		
		Z mnohých výsledkov a zistení tejto práce sú pre túto bakalársku prácu najrelevantnejšie zistenia týkajúce sa motivácie hráčov počas zápasu v závislosti od jeho vývoja. Pri interpretácií výsledkov je dôležité brať do úvahy bodovanie výsledného skóre futbalového zápasu a medzné straty, resp. zisky tímov vzťahujúce sa na jednotlivé výsledky, viď tabuľka \ref{table:futbal;}. 
		
		\begin{table}[h]
			\catcode`\-=12
			\centering	
			\bgroup
			\def\arraystretch{1.2}
			\begin{tabular}{|c|c|c|c|c|}
				\hline
				
				\textbf{Rozdiel v skóre} & \textbf{Stav zápasu} & \textbf{Body} & \textbf{Medzná strata} & \textbf{Medzný zisk} \\ \hline \hline
				$ \leq $ -3 & \multirow{3}{*}{prehra} & 0 & 0 & 0 \\
				-2 &  & 0 & 0 & 0 \\
				-1 &  & 0 & 1 & 0 \\ \hline \hline
				0 & remíza & 1 & 2 & 1 \\ \hline \hline
				1 & \multirow{3}{*}{výhra} & 3 & 0 & 2 \\
				2 &  & 3 & 0 & 0 \\
				$ \geq $ 3 &  & 3 & 0 & 0 \\ \hline				
			\end{tabular}
			\egroup
			\caption{Bodovanie výsledného skóre futbalového zápasu, medzné straty a zisky}
			\vskip\abovecaptionskip\emph{Zdroj: <<\cite{schneemann2017} a vlastné spracovanie>>}
			\label{table:futbal;}
		\end{table}
		
		Prvý stĺpec tabuľky \ref{table:futbal;} obsahuje hodnoty rozdielu v skóre medzi dvomi tímami. Druhý stĺpec priraďuje danému rozdielu v skóre pomenovanie stavu. Počet bodov, ktoré tím získa v súvislosti s niektorým zo stavov zápasu sa nachádza v treťom stĺpci. Posledné dva stĺpce vypovedajú o medznej strate a medznom zisku ak sa stav zmení. To, o čom vypovedá táto tabuľka možno ilustrovať na príklad: ak tím prehráva o dva góly, jeho bodový zisk je nula. Ak by dal jeden gól, prehrával by síce už iba o rozdiel jedného gólu, avšak jeho bodový zisk by bol stále nula. Tým pádom by sa nezmenila hodnota ani medznej straty ani zisku. Ak by však dal ešte jeden gól, skóre by sa vyrovnalo a stav by sa zmenil na remízu. Bodový zisk tímu by bol jedna. Medzný zisk tohto posunu z prehry na remízu by bol jedna. Ak by dal tím ešte tretí gól, stav by sa zmenil na výhru, bodový zisk by bol tri body a medzný zisk oproti predchádzajúcej pozícií by bol dva.
		
		Najväčšia motivácia sa prejavila v prípade, keď tím viedol o jeden gól. Akonáhle tím zaostával za súperom o jeden gól, motivácia bola v porovnaní s remízovým skóre alebo vedúcim skóre podstatne nižšia. Autori tento fenomén opodstatňujú hráčskou averziou voči prehre, tzn. že oveľa viac im záleží na tom, aby sa vyhli prehre ako na tom, aby získali výhru. 
		
		V prípade, že tím A vedie v zápase jednogólovým rozdielom, stačí jeho súperovi, tímu B, jediný gól na to, aby sa stav zmenil na remízu. Doteraz vyhrávajúci tím, by miesto troch bodov získal iba jeden, tzn. že by prišiel o dva body. Možnosť straty dvoch bodov je pre tím A výraznejším podnetom, akým je pre tím B možnosť získať dva body. 
	
		Rovnaký princíp je uplatniteľný aj pri porovnávaní remízy a prehrávania o jeden gól. Tímy sa viac obávajú toho, že sa pre nich stav zápasu zmení z remízy na prehru (a miesto jedného bodu by nezískali žiadny) ako si cenia zmenu stavu z prehry na remízu (miesto žiadneho bodu by získali jeden). \parencite[s.~1768]{schneemann2017}
		
		V porovnaní s prácou Bergera a Popa (\citeyear{berger2011}) treba brať do úvahy rozdielnosť povahy skúmaných športov a to najmä v súvislosti s množstvom bodov, ktoré počas jednotlivých zápasov padnú. Skórovanie v basketbale je v porovnaní s futbalom, kde často rozhodne jediný gól, oveľa frekventovanejšie.
		
		\subsection{Barankay}
		Rozdielne výsledky priniesla aj štúdia Barankaya (\citeyear{barankay2010}), ktorá sa nezameriava na oblasť športu a sústreďuje sa na motivačný efekt zamestnancov. 
		
		Autor štúdie skúma do akej miery vie porovnávanie sa s druhými ovplyvniť ľudské správanie a teda zvýšiť alebo znížiť snahu a výkon. Cieľom tzv. field experimentu bolo zistiť, ako ľudia prispôsobia svoju snahu v prípade, že dostanú hodnotenie svojej práce porovnávajúce ich s ostatnými zamestnancami. 
		
		Účastníci tohto experimentu, boli najatí na prácu analýzy obrázkov. Zamestnanci boli následne rozdelení do dvoch skupín, pričom členovia experimentálnej skupiny dostali spätnú väzbu o tom, aké je ich hodnotenie voči ostatným v otázke presnosti ich práce. 
		
		Výsledkom experimentálnej skupiny oproti kontrolnej skupine bola 30\% šanca, že sa zamestnanci do práce nevrátia, a ak sa vrátia, sú o 22\% menej výkonní. To znamená, že závery tejto štúdie poukazujú na fakt, že pri „prehrávaní“ nedochádza k zlepšeniu, práve naopak, prevláda demotivácia. \parencite[s.~4]{barankay2010}
		
		V tomto experimente však hodnotenia nemali vplyv na finančné ohodnotenie zamestnancov. Pri športových zápasoch by sa dalo argumentovať, že vzájomné porovnávanie dvoch tímov počas zápasu môže ovplyvniť motiváciu tímu inak ako v tomto prípade. Nutný je však predpoklad, že lepšie výsledky tímu a viac výhier môžu znamenať viac peňazí pre tím a teda lepšie finančné ohodnotenie jednotlivých hráčov.
		
		\subsection{Eriksson, Poulsen, Villeval}
		Štúdia Erikssona, Poulsena, Villevala (\citeyear{eriksson2009}) experimentálne testuje, mimo iné, ako vplýva spätná väzba týkajúca sa relatívneho výkonu zamestnancov na ich snahu a motiváciu. Skúmané sú rôzne formy podania spätnej väzby, napr. v polovici výrobného obdobia alebo priebežne. Autori sa snažia rozhodnúť, aký vplyv má relatívne výkonnostné hodnotenie zamestnancov, ktoré sú často využívané firmami, na snahu a motivovanosť.
		
		Ak je zamestnanec finančne hodnotený na základe porovnania jeho výkonu s výkonmi ostatných zamestnancov, naskytá sa otázka, či zamestnanca informovať o tom, ako si vedie a ak áno, v akej frekvencii. Možnosti vplyvu sú dve: keď zamestnanec zistí, že je pozadu a horší, buď ho to znechutí a odradí od ďalšej práce alebo vďaka tejto informácií bude pracovať tvrdšie, aby sa vyhol prípadnému zahanbeniu. \parencite[s.~679]{eriksson2009}
		
		Zistenia práce hovoria, že uverejnenie priebežnej spätnej väzby relatívneho výkonu nezvyšuje a nezlepšuje výkonnosť zamestnancov. Tí zamestnanci, ktorí boli popredu sa síce nezačnú snažiť výrazne menej, avšak tí pozadu sa zhoršujú. Nie preto, že by začali vynakladať menšie množstvo snahy, ale kvôli nižšej kvalite odvádzanej práce. Pod nižšou kvalitou je možné si predstaviť napríklad väčšie množstvo chýb pri práci, ktoré môžu byť spôsobené stresom a tlakom vytvoreným práve zverejnenou spätnou väzbou. \parencite[s.~687]{eriksson2009}
		
		Za zmienku stojí aj zistenie, že ak zamestnanci sami porovnávajú svoje výkony s výkonmi ostatných, tí, ktorí si vedú horšie sa nevzdávajú a to aj vtedy keď medzera, ktorá ich delí od tých, čo sú popredu je väčšia a o víťazovi je už v podstate rozhodnuté. \parencite[s.~686]{eriksson2009}
		
				
	\chapter{Basketbal ako šport}
	Cieľom tejto kapitoly nie je ísť do podrobností v súvislosti s pravidlami alebo históriou jednotlivých líg, ale v stručnosti definovať basketbal ako šport a jednotlivé basketbalové ligy, ktoré sú zdrojom dát pre túto prácu.
	
	Basketbal je kolektívny šport desiatich hráčov rozdelených do dvoch súperiacich tímov. Cieľom hry rozdelenej na štyri štvrtiny je získať viac bodov ako súper. Body sú udeľované za trafenie lopty do basketbalového koša druhého tímu. Počet bodov za jednotlivé skórovania sa líši podľa ich povahy a miesta, z ktorého bol kôš trafený. Kôš hodený z dvojbodového územia je ohodnotený dvomi bodmi, kôš zo vzdialenejšieho územia je za tri body a trestný hod je za jeden bod. Zápasy pritom nemôžu skončiť remízou, podobne ako je to napríklad pri tenise. V prípade, že sa o víťazovi zápasu nerozhodne v základnej hracej dobe, nastavuje sa toľko predĺžení, koľko je potrebné.
	
		\section{Charakteristika basketbalových líg}
		Táto bakalárska práca pracuje so zápasmi z nasledujúcich piatich basketbalových líg:
		\begin{itemize}
		\item Národná basketbalová liga (ČR),
		\item Ženská basketbalová liga (ČR),
		\item Národná basketbalová asociácia (USA),
		\item Ženská národná basketbalová asociácia (USA),
		\item Euroliga.
		\end{itemize}
	
		Prvé dve ligy sú českými najvyššími basketbalovými ligami. Národná basketbalová liga, skrátene NBL, je mužská liga, ktorá vznikla v roku 1993 po rozdelení Československa a zániku Československej basketbalovej ligy. Počet basketbalových tímov je dvanásť.
		
		Ženská basketbalová liga, skrátene ŽBL, je najvyššou českou ženskou basketbalovou ligou a 
		vznikla v roku 1993 za rovnakých okolností ako NBL. Súčasný názov však nesie až od roku 2005, premenovaná bola z 1. basketbalovej ligy žien. Počet jej tímov sa v priebehu rokov menil, v súčasnosti je ustálený na rovnakom čísle ako má NBL.
		
		Americká Národná basketbalová asociácia, známa pod skratkou NBA, je najpopulárnejšou basketbalovou ligou sveta, ktorá má aj spomedzi tu spomenutých líg najdlhšiu históriu nakoľko vznikla v roku 1946. Z organizačného hľadiska sa delí na Východnú a Západnú konferenciu. Každá z nich sa ďalej delí na tri divízie, ktoré majú po päť tímov, takže NBA tvorí dokopy tridsať basketbalových klubov.
		
		Ženská národná basketbalová asociácia, WNBA, je ženským náprotivkom mužskej NBA, ktorá sa líši napríklad dĺžkou existencie, počtom tímov či minutážou štvrtín – WNBA bola založená v roku 1994, tvorí ju dvanásť tímov a štvrtiny netrvajú dvanásť, ale desať minút.
		
		Poslednou ligou je Euroliga, ktorá je európskou basketbalovou mužskou ligou, tvorenou niekoľkými víťazmi národných líg a niektorými vybranými európskymi tímami – dokopy ide o šestnásť klubov. 
		
		Každá liga má svoje špecifiká. Rozdielov v pravidlách medzi jednotlivými ligami je niekoľko, pre túto prácu však nie sú významné. Spomenúť však možno napríklad veľkosť ihriska, veľkosť dvojbodového územia či dĺžku prestávky medzi prvou a druhou, resp. treťou a štvrtou štvrtinou. Okrem toho sa môže líšiť aj systém súťaže. Pre ilustráciu, systém súťaže českej NBL a ŽBL sa skladá z troch nasledovných časti: základnej časti, play off a play out. Základná časť, alebo aj tzv. dlhodobá časť, je dvojkolová hra každého tímu s každým – na domácom, resp. súperovom ihrisku. 
		
		Play off sa skladá zo štvrťfinále, semifinále, finále a zápasu o tretie miesto. Štvrťfinále sa hrá na základe umiestnenia tímov v základnej časti podľa vzorca prvý s ôsmim, druhý so siedmim, tretí so šiestym a štvrtý s piatym. Štvrťfinále sa hrá na tri víťazné zápasy, takže maximálny počet zápasov medzi dvomi tímami je päť. Do semifinále postupujú štyri víťazné tímy, ktoré hrajú podľa obdobného vzorca ako vo štvrťfinále, tzn. prvý so štvrtým a druhý s tretím a to na základe umiestnenia zo základnej časti. Opäť sa hrá na tri víťazstvá. Víťazi zo semifinále postupujú do finále a kto vyhrá opätovne ako prvý tri zápasy, získa titul majstra Českej republiky.
		
		Play out je časť súťaže, kedy posledné štyri tímy, teda tie, ktoré sa nedostali do play off, hrajú tzv. kvalifikačnú skupinu formou každý s každým o udržanie sa v NBL. Tím, ktorý sa umiestni na poslednom mieste zostupuje do nižšej ligy, tj. 1.ligy. Na jeho miesto postúpi víťaz 1.ligy, ktorý však môže odmietnuť a v tom prípade je podľa pravidiel táto možnosť postúpená tímu, ktorý skončil na druhom mieste. \parencite{bulletin2018}

	\chapter{Model regresnej diskontinuity a jeho využitie}
		\section{Regresná diskontinuita}
		\label{sec:rdd}
		Použitou metódou v tejto práci je tzv. regression discontinuity design, v preklade model regresnej diskontinuity alebo aj nespojitá regresia, známa pod skratkou RDD. Ako prví predstavili regresnú diskontinuitu autori Thistlethwaite a Campbell (\citeyear{thist1960}), ktorí vo svojej práci skúmali vplyv odmeny za dobré štúdium na budúce akademické úspechy študentov v Spojených štátoch. Dobrým štúdiom je myslený dobrý výsledok v teste PSAT~\footnote{Plný angl. názov Preliminary SAT, známy aj ako National Merit Scholarship Qualifying Test.}, ktorý píšu študenti stredných škôl a na základe ktorého sú následne udeľované štipendiá organizáciou NMSC~\footnote{Plný angl. názov National Merit Scholarship Corporation.}. Budúce akademické úspechy predstavuje tzv. GPA~\footnote{Plný angl. názov Grade Point Average je spôsob merania akademickej úspešnosti známy najmä v USA.}.
	
		Skúmanie prebiehalo nasledovne: odmeny vo forme štipendií sú udeľované pri dosiahnutí konkrétneho počtu bodov, tzn. že existuje určitá hranica, ktorá keď je prekonaná, tak je študentovi udelené štipendium. Naopak, ak študent bodovú hranicu nedosiahne, na štipendium nemá nárok. 
		
		Študenti, ktorí dosiahli bodové ohodnotenie tesne nad a tesne pod hranicou sú rovnakí, alebo aspoň veľmi podobní, a v podstate dosiahli rovnaké výsledky, avšak tí pod hranicou štipendium nezískajú a tí nad áno. Model regresnej diskontinuity porovnáva v práci Thistlethwaita a Campbella (\citeyear{thist1960}) študentov tesne pod a tesne nad hranicou a rozdiel medzi nimi pripisuje vplyvu udelenia štipendia. 
	
		Očakávania autorov sa v tomto prípade potvrdili. S vyšším počtom získaných bodov  v teste PSAT mali študenti lepšie GPA, avšak vďaka pôsobeniu štipendia sa v hraničnom bode pre zisk štipendia vyskytol nespojitý skok. Tento skok a jeho veľkosť predstavuje odhad efektu regresnej diskontinuity.
		
		Vo svojej najjednoduchšej podobe má model regresnej diskontinuity nasledujúcu podobu:
		
		\begin{equation}
		Y_{i} = \alpha + \beta _{1} [\textit{X}]_{i} + \beta _{2} [\textit{D}]_{i} + \varepsilon_{i}
		\end{equation}
		
		kde 
		
		 \begin{equation}
		D = \begin{cases}
		1 & \text{ak }x\geq{h}, \\
		0 & \text{ak }x < {h}.
		\end{cases}
		\end{equation}
		
		\textit{Y} je vysvetľovaná premenná, $\alpha$ je úrovňová konštanta, premenná \textit{X} je hodnota vysvetľujúcej premennej, \textit{D} je umelá premenná, ktorá vyjadruje či je \textit{X} nad alebo pod hranicou h a $\varepsilon_{i}$ je náhodná zložka a $\beta _{1}$ a $\beta _{2}$ sú parametre vysvetľujúcich premenných. 
		
		To znamená, že pri bližšom pohľade na príklad so štipendiami je hodnota GPA vysvetľovanou premennou \textit{Y}, výsledok v teste PSAT je \textit{X} a \textit{D} nadobúda hodnotu jedna ak je študent nad bodovou hranicou a hodnota nula ak je pod ňou.
	
		Ďalším príkladom práce, v ktorej autori využili RDD, je napríklad práca Carpentera a Dobkina (\citeyear{carpenter2011}), ktorí skúmali minimálnu legálnu hranicu pitia alkoholu v USA a jej vplyv na konzumáciu alkoholu mladých ľudí. Banks a Mazzonna (\citeyear{banks2012}) sa zamerali na zvýšenie minimálnej školskej dochádzky zo štrnásť na pätnásť rokov a skúmali vplyv dodatočného roku štúdia na kognitívne schopnosti v staršom veku. Regresnú diskontinuitu používa ako osvedčenú a efektívnu metódu aj Európska komisia, ktorá ju využíva ako metódu hodnotenia štátnej pomoci. \parencite[s.~20]{ek2014} 
		
		\section{Dataset}
		Využité dáta v tejto práci pochádzajú z databáze športových výsledkov a kurzov Trefík, odkiaľ boli stiahnuté dostupné basketbalové dáta z piatich basketbalových líg – NBL, ŽBL, NBA, WNBA a Euroligy. \parencite{trefik}
		
		Po stiahnutí dát bolo potrebné ich upravenie a prefiltrovanie na základe predmetu skúmania a potreby nášho modelu a jeho variácií, keďže dáta obsahovali nadbytočné informácie alebo nemali vhodnú formu. Následné úpravy sa týkali všetkých datasetov. V prvom rade boli zo všetkých datasetov odstránené tie zápasy, ku ktorým neboli dostupné informácie o bodovom stave v polčase, keďže polčasový stav je pre zvolený model kľúčový. Taktiež boli vyradené všetky zápasy ktoré sa skončili remízou, nakoľko táto bakalárska práca sa nebude zaujímať o výsledky po základnej hracej dobe. 
		
		Následne boli odstránené nepotrebné dáta relevantných zápasov, ktoré nie sú pre túto prácu potrebné, ako napr. dátum zápasu, výsledný bodový stav zápasu či bodový stav za tretiu a štvrtú štvrtinu.
		
		Dáta obsahovali informáciu o víťazovi zápasu, pričom hodnota jedna značila výhru domáceho tímu, hodnota dva jeho prehru, resp. výhru hosťujúceho tímu. Hodnoty dva boli prekódované na hodnoty nula. Tak vznikla premenná \textit{WIN}.
		
		Ďalej bola vytvorená premennú predstavujúca bodový rozdiel medzi domácim a hosťujúcim tímom v polčase –  \textit{SCORE\_HALF}. Táto premenná nebola priamo dostupná v stiahnutých dátach, ale bolo ju možné  odvodiť od  bodového skóre domáceho a hosťujúceho tímu v polčase. Táto informácia bola získaná spočítaním počtu bodov za prvú a druhú štvrtinu. To znamená, že najprv boli sčítané body za prvú a druhú štvrtinu pre domáci a hosťujúci tím, z čoho vznikli body v polčase pre oba tímy a v náväznosti na to boli rozdielom týchto bodov získané hodnoty pre premennú \textit{SCORE\_HALF}. 
		
		Nutnosťou bolo vytvoriť z dát bodového rozdielu v polčase novú premennú \textit{HALF}, v ktorej bola výhra domáceho tímu po polčase zakódovaná hodnotou nula a prehra hodnotou~jedna. Treba poznamenať, že zápasy, ktoré sa skončili v polčase remízou neboli z datasetu vylúčené a boli kódované ako výhra, nakoľko nás zaujímala zmena šance na výhru, keď sa domáci tím nachádzal striktne pozadu.
		
		Poslednou úpravou dát bolo odčítanie hodnoty jedna od dát predstavujúcich výherný kurz na víťazstvo domáceho tímu. Zdôvodnenie sa nachádza v časti \ref{sec:pouzitymodel}.
	
		\begin{table}[t]
			\catcode`\-=12
			\centering	
			\bgroup
			\def\arraystretch{1.3}
			\begin{tabular}{|c|c||c|c|c|c|}
				\hline
					
				\multirow{2}{*}[-1em]{\makecell {\textbf{Basketbalová} \\ \textbf{liga}}}
				& \multirow{2}{*}[-1em]{\textbf{Obdobie}}
				& \multicolumn{4}{c|}{\large\textbf{Zápasy}} \\ \cline{3-6}
									
				& & \makecell{\textbf{Pôvodný} \\ \textbf{počet s dátami}} &  \makecell{\textbf{Remíza} \\ \textbf{v polčase}} & \makecell{\textbf{Remíza} \\ \textbf{na konci}} & \makecell{\textbf{Použitý} \\ \textbf{počet}} \\ \hline \hline
					
				\textit{ČBF} & 2008-2015 & 1 371 & 38 & 46 & 1 325 \\ \hline
				\textit{ŽBL} & 2011-2015 & 556 & 17 & 5 & 551 \\ \hline \hline
					
				\textit{NBA} & 2002-2015 & 15 979 & 569 & 1 009 & 14 970 \\ \hline
				\textit{WNBA} & 2007-2014 & 1 785 & 55 & 135 & 1 650 \\ \hline \hline
					
				\textit{Euroliga} & 2002-2015 & 2 755 & 92 & 128 & 2 627 \\ \hline
			\end{tabular}
			\egroup
			\caption{Prehľad filtrácie dát}
			\vskip\abovecaptionskip\emph{Zdroj: <<vlastné spracovanie>>}
			\label{table:prefiltrovane-data;}
		\end{table}
	
		Po všetkých vyššie spomenutých úpravách sme mali na ekonometrickú analýzu vhodných dovedna viac ako dvadsať tisíc zápasov. Tabuľka \ref{table:prefiltrovane-data;} poskytuje prehľadný súhrn týchto zápasov, ich príslušnosť k jednotlivým ligám, či rámcové obdobie, z ktorého dané zápasy sú. Stĺpec \textit{Pôvodný počet s dátami} reprezentuje počet zápasov, ktoré obsahovali všetky nevyhnutné informácie potrebné pre zostrojenie ekonometrického modelu. Stĺpec \textit{Remíza v polčase} zobrazuje počet zápasov, ktoré boli v polčase v stave remízy a boli ponechané v datasete pod kódom výhry. Nasledujúci stĺpec, \textit{Remíza na konci} je počet zápasov, ktoré sa v základnej hracej dobe skončili s rovnakým bodovým skóre pre domáci a hosťujúci tím a preto boli z datasetu vylúčené. Posledný stĺpec je súhrnom toho, koľko zápasov bolo pre tú ktorú basketbalovú ligu nakoniec dostupných. Najväčšiu vzorku pozorovaní predstavuje americká NBA s 14 970 zápasmi, zatiaľčo najmenšou vzorkou je česká ŽBL s počtom 551.
		
		Tak ako bolo už niekoľko krát spomenuté, v tejto práci je porovnávaný vždy domáci a hosťujúci tím. Pri každom zápase je však skúmaný iba jeden z týchto tímov, aby sa vyhlo duplicite. V zásade je bezpredmetné, či by bol vybraný domáci alebo hosťujúci tím, nakoľko hociktorý jeden je zrkadlovým odrazom toho druhého. Výber domáceho tímu zhodný s výberom Bergera a Popa (\citeyear{berger2011}), ktorí sa preň rozhodli tiež. Náhodný výber tímu by nebol v tomto prípade vhodnou metódou, keďže by mohol viesť k rozdielnym výsledkom v závislosti na náhodnosti výberu.
		
		Testovanie následne prebiehalo postupne v rámci jednotlivých líg, čo nám dalo možnosť nielen skúmať rozdiely medzi národnými ligami, ale napríklad aj rozdiely v motivácií v mužských a ženských ligách.
		
		\section{Použitý ekonometrický model}
		\label{sec:pouzitymodel}
		Pri zostavovaní ekonometrického modelu sa vychádzalo zo spomínanej štúdie Bergera a Popa (\citeyear{berger2011}), ktorý bol pozmenený v súvislosti s cieľmi tejto práce. Dáta boli spracované vo voľne prístupnom programe \textit{R Studio} a mali podobu prierezových dát.
		
		Podoba nášho základného ekonometrického modelu je nasledovná:
		\begin{multline}
		WIN_{i} = \alpha + \beta _{1} [\textit{HALF}]_{i} + \beta _{2} [\textit{SCORE~\_~HALF}]_{i} + \beta _{3} [\textit{SCORE~\_~HALF2}]_{i} \\
		+ \beta _{4} [\textit{SCORE~\_~HALF3}]_{i} + \beta _{5} [\textit{BETTING\_RATE}]_{i} + \varepsilon_{i},
		\end{multline}
		
		Vysvetľovanou premennou je premenná \textit{WIN}, ktorá je umelou premennou nadobúdajúcou hodnotu jedna ak domáci tím vyhrá zápas a nula ak tento zápas prehrá, resp. zápas vyhrá súper. 
		
		Vysvetľujúcich premenných je dovedna päť. Prvou je umelá premenná \textit{HALF}, ktorá udáva informáciu o tom, či domáci tím v polčase prehrával alebo vyhrával. Ak vyhrával, premenná nadobúda hodnotu nula a ak prehrával, hodnota je jedna. Zdôvodnením je, že túto prácu zaujímajú prípady zápasov, kedy domáci tím v polčase prehrával a predpokladáme, že malá bodová strata na súpera ho motivovala k lepšiemu výkonu a teda zdolaniu súpera a víťazstvu. 
		
		Ďalšou vysvetľujúcou premennou je premenná \textit{SCORE\_HALF}, ktorá udáva bodový rozdiel v polčase medzi dvomi tímami. Premenné \textit{SCORE\_HALF2} a \textit{SCORE\_HALF3} sú umocnením premennej \textit{SCORE\_HALF} na druhú, resp. na tretiu. Dôvodom zahrnutia týchto dvoch premenných je, že pravdepodobnosť výhry nemusí byť lineárna v tom, koľko domáci tím prehráva v polčase. S druhým a tretím umocnením premennej \textit{SCORE\_HALF} je možné lepšie prispôsobiť naše dáta tomu, aby bolo možné vidieť skok regresnej diskontinuity, a to za predpokladu, že sa tam nachádza.
		
		\textit{BETTING\_RATE} je poslednou vysvetľujúcou premennou, ktorá reprezentuje, ako ovplyvní model to, či je domáci tím favoritom. Čím vyššia je premenná, tým vyšší je kurz na víťazstvo domáceho tímu, čo znamená, že tím je menším favoritom. Aj naopak platí, že čím nižší kurz, tým je tím väčším favoritom. Odčítanie hodnoty jedna od kurzu na víťazstvo domáceho tímu nie je nutnosťou, avšak v tejto podobe dáva premenná informáciu o čistom zisku, napr. z jednej koruny a následná interpretácia tak môže byť jednoduchšia. To znamená, že keď sa vsadí na výhru tímu desať eur pri kurze je 1,5, prípadná výhra bude predstavovať 15,40 eur, z ktorých desať sa vráti ako pôvodná investícia a päť eur je čistý zisk.		
		
		Okrem toho model obsahuje úrovňovú konštantu $\alpha$, parametre vysvetľujúcich premenných $\beta _{1}$ až $\beta _{5}$  a náhodnú zložku $\varepsilon_{i}$. Prehľad jednotlivých premenných a ich popis obsahuje tabuľka \ref{table:premenne;}.
	
		\begin{table}[t]
			\catcode`\-=12
			\centering	
			\bgroup
			\def\arraystretch{1.3}
			\begin{tabular}{l l}
				\hline
				\textbf{Premenná} & \textbf{Popis} \\ \hline \hline
				\textit{WIN} & výhra, resp. prehra domáceho tímu \\
				\textit{HALF} & výhra, resp. prehra domáceho tímu v polčase \\
				\textit{SCORE\_HALF} &  rozdiel v polčase medzi domácim a hosťujúcim tímom \\
				\textit{SCORE\_HALF2} & rozdiel v polčase umocnený na druhú \\
				\textit{SCORE\_HALF3} & rozdiel v polčase umocnený na tretiu \\
				\textit{BETTING\_RATE} & kurz na víťazstvo domáceho tímu mínus jedna \\ \hline
			\end{tabular}
			\egroup
			\caption{Stručný prehľad vysvetľovanej premennej a vysvetľujúcich premenných}
			\vskip\abovecaptionskip\emph{Zdroj: <<vlastné spracovanie>>}
			\label{table:premenne;}
		\end{table}
	
	\chapter{Hlavné modely a výsledky}
	Táto bakalárska práca si stanovila nasledovné tri ciele: overiť robustnosť záveru štúdie Bergera a Popa (2011), z hľadiska existencie a pôsobenia motivačného efektu porovnať mužský a ženský basketbal a tiež skupinu favoritov a tých ostatných. Na dosiahnutie týchto cieľov bolo skonštruovaných niekoľko alternatív základného modelu bližšie popísaného v podkapitole \ref{sec:pouzitymodel}.
	
	Zostrojené grafy \ref{fig:ZBL4}, \ref{fig:NBA4} a \ref{fig:NBL4}, \ref{fig:WNBA4}, \ref{fig:EURO4} sú grafmi regresnej diskontinuity a zobrazujú výsledky pre všetkých päť skúmaných líg. Každý graf predstavuje percento vyhraných zápasov domáceho tímu v závislosti na bodovom rozdiele medzi domácim a hosťujúcim tímom v polčase. Modrá čiara zobrazuje lineárny tvar dát. Čierna vertikálna čiara v hodnote nulového bodového rozdielu predstavuje miesto, v ktorom je očakávaná diskontinuita v prípade existencie motivačného efektu.

	Tabuľky \ref{fig:ZBL4}, \ref{fig:NBA4} a \ref{fig:NBL4}, \ref{fig:WNBA4}, \ref{fig:EURO4} sú výsledkom ekonometrického modelu logitu. Každá tabuľka zobrazuje vplyv prehrávania v polčase na celkovú výhru v zápase pre všetkých päť basketbalových líg v poradí NBL, WNBA, EUROLIGA, ŽBL a NBA. Vysvetľovanou premennou je premenná \textit{WIN}. V prvom stĺpci zľava sú jednotlivé vysvetľujúce premenné – \textit{HALF}, \textit{SCORE\_HALF}, \textit{SCORE\_HALF2}, \textit{SCORE\_HALF3} a \textit{BETTING\_RATE}. 

	Stĺpce s označením (1), (2), (3) a (4) sú štyrmi rôznymi variantami základného modelu. Modely v stĺpcoch (1) a (3) obsahujú premennú \textit{SCORE\_HALF} iba v jej lineárnej podobe, zatiaľ čo modely (2) a (4) obsahujú aj jej kvadratickú a kubickú podobu. Ďalej modely (1) a (2) neobsahujú žiadnu ďalšiu kontrolnú premennú, modely (3) a (4) však obsahujú aj kontrolnú premennú \textit{BETTING\_RATE}. 
	
	Hodnoty v tabuľke sú výsledky odhadov koeficientov logit modelu, v zátvorke sú uvedené získané smerodatné odchýlky odhadu jednotlivých parametrov. Počet hviezdičiek symbolizuje štatistickú významnosť, kde * značí významnosť na hladine 10 \%, ** na hladine 5 \% a *** na hladine 1\%.

	\section{Overenie robustnosti}
	\label{sec:overenie}
	Pre účely tejto práce budú výsledky štúdie Bergera a Popa (\citeyear{berger2011}) označené za robustné vtedy, keď sa rovnaké výsledky objavia aj v ďalších ligách, v menších a väčších datasetov, u mužov aj u žien. To znamená, že je potrebné aby sa v dátach prejavila diskontinuita a to v podobe skoku v bode nulového bodového rozdielu v polčase. Takýto skok je viditeľný v grafe \ref{fig:ZBL4} a v grafe \ref{fig:NBA4}, ktoré prislúchajú ligám ŽBL a NBA. Dáta ostatných troch líg takouto diskontinuitou nedisponujú, viď grafy \ref{fig:NBL4}, \ref{fig:WNBA4}, \ref{fig:EURO4}\footnote{Grafy \ref{fig:NBL4}, \ref{fig:WNBA4}, \ref{fig:EURO4} sú dostupné v prílohe A.}.
		
	Pri pohľade na tabuľky \ref{table:vplyvNBL} až \ref{table:vplyvZBL} a \ref{table:vplyvNBAja}, ktoré ukazujú vplyv prehrávania v polčase na výhru nás zaujímajú najmä dve veci a to v súvislosti s premennou \textit{HALF} – štatistická významnosť a znamienko koeficientu premennej \textit{HALF}. 
	
	Štatistická významnosť modelu vypovedá o tom, že s nejakou rozumnou mierou pravdepodobnosti\footnote{Typicky 5 \% mierou istoty pri 95 \% intervale spoľahlivosti.} nie je efekt nulový a existuje. To znamená, že akonáhle je model štatisticky nevýznamný, efekt väčšej motivácie pri bytí pozadu neexistuje.
	
	Znamienko koeficientu, kladné alebo záporné, premennej \textit{HALF}, hovorí o smere vplyvu motivačného efektu. Ak je znamienko kladné, byť trochu pozadu v polčase znamená väčšiu pravdepodobnosť výhry, zatiaľ čo záporné znamienko naznačuje opak, a to, že byť pozadu v polčase znamená väčšiu pravdepodobnosť prehry.
	
	\begin{table}[h]
		\catcode`\-=12
		\centering 
		\bgroup
		\def\arraystretch{1.5}
		\begin{tabular}{|r||c|c|}
			\hline
			& Štatistická významnosť & Znamienko \tabularnewline \hline \hline
			
			\textit{NBL} & N & + \tabularnewline
			\hline 
			\textit{ŽBL} & A & - \tabularnewline
			\hline 
			\textit{NBA} & A & + \tabularnewline
			\hline 
			\textit{WNBA} & N & +/- \tabularnewline
			\hline 
			\textit{EUROLIGA} & N & +/- \tabularnewline
			\hline 
		\end{tabular}
		\egroup
		\caption{Znamienka a štatistická významnosť premennej \textit{HALF} pre všetky ligy}
		\vskip\abovecaptionskip\emph{Zdroj: <<vlastné spracovanie>>}
		\label{table:prehladznamienok}
	\end{table}	
	
	Tabuľka \ref{table:prehladznamienok} sumarizuje štatistickú významnosť a znamienko premennej \textit{HALF} pre modely všetkých piatich basketbalových líg. V prvom stĺpci sú zoradené skúmané basketbalové ligy. V druhom stĺpci sa nachádza vyjadrený výskyt štatistickej významnosti. Písmeno A značí, že premenná je štatisticky významná, písmeno N, že nie je. Tretí stĺpec obsahuje symbol podľa toho, či je premenná kladná alebo záporná, kombinácia symbolu plus a mínus predstavuje nejednoznačnosť, kedy časť modelov vyšla kladná a časť záporná, avšak hodnoty samy o sebe boli v okolí nuly.
	
	Vo všeobecnosti je možné skonštatovať, že vo všetkých piatich modeloch platí, že čím viac sú tímy v polčase popredu, tým viac sa zvyšuje ich šanca na výhru v zápase a naopak, čím väčšia je ich strata, tým je táto šanca nižšia. To je vidieť nielen na grafoch ale aj v tabuľkách, v ktorých kladné znamienka premennej \textit{SCORE\_HALF} hovoria, že čím je vyšší bodový rozdiel v polčase, tým sa zvyšuje pravdepodobnosť výhry, záporné symbolizujú opačný efekt.
	
	Záporné znamienka koeficientov premennej \textit{BETTING\_RATE} zas vypovedajú o tom, že čím je vyšší kurz na basketbalový tím, tým je nižšia pravdepodobnosť výhry v zápase. Platí, že čím má tím vyšší kurz, tím menším favoritom v zápase je.
	
	\begin{table}[h]
		\catcode`\-=12
		\centering 
		\footnotesize
		\bgroup
		\def\arraystretch{1.5}
		\begin{tabular}{|r||x{\sizeofcolumn}x{\sizeofcolumn}x{\sizeofcolumn}x{\sizeofcolumn}|}
			\hline
			
			\multirow{2}{*}{} & \multicolumn{4}{c|}{Závislá premenná: $ WIN_{i} $ = 1 ak domáci tím vyhral zápas} \tabularnewline \cline{2-5}
			& \textbf{(1)} & \textbf{(2)} & \textbf{(3)} & \textbf{(4)} \tabularnewline \hline \hline
			
			\multirow{2}{*}{\textit{HALF}} & 0,342 & 0,046 & 0,405 & 0,198 \tabularnewline
			& (0,244) & (0,305) & (0,259) & (0,319) \tabularnewline
			\hline
			
			\multirow{2}{*}{\textit{SCORE\_HALF}} & 0,203*** & 0,150*** & 0,191*** & 0,153*** \tabularnewline
			& (0,017) & (0,034) & (0,019) & (0,035) \tabularnewline
			\hline
			
			\multirow{2}{*}{\textit{SCORE\_HALF2}} & --- & -0,003* & --- & -0,002 \tabularnewline
			& & (0,001) & & (0,001) \tabularnewline
			\hline
			
			\multirow{2}{*}{\textit{SCORE\_HALF3}} & --- & 0,0003* & --- & 0,0002 \tabularnewline
			& & (0,0002) & & (0,0002) \tabularnewline
			\hline
			
			\multirow{2}{*}{\textit{BETTING\_RATE}} & --- & --- & -0,555*** & -0,547*** \tabularnewline
			& & & (0,071) & (0,071) \tabularnewline
			\hline
			
			$ Pseudo-R^{2} $ & 0,356 & 0,360 & 0,413 & 0,415 \tabularnewline
			\hline
			
			\textit{Pozorovania} & 1 325 & 1 325 & 1 325 & 1 325 \tabularnewline
			\hline
		\end{tabular}
		\egroup
		\caption{Vplyv prehrávania v polčase na výhru v NBL dátach}
		\vskip\abovecaptionskip\emph{Zdroj: <<vlastné spracovanie>>}
		\label{table:vplyvNBL}
	\end{table}
	
	\begin{table}[h]
		\catcode`\-=12
		\centering 
		\footnotesize
		\bgroup
		\def\arraystretch{1.5}
		\begin{tabular}{|r||x{\sizeofcolumn}x{\sizeofcolumn}x{\sizeofcolumn}x{\sizeofcolumn}|}
			\hline
			
			\multirow{2}{*}{} & \multicolumn{4}{c|}{Závislá premenná: $ WIN_{i} $ = 1 ak domáci tím vyhral zápas} \tabularnewline \cline{2-5}
			& \textbf{(1)} & \textbf{(2)} & \textbf{(3)} & \textbf{(4)} \tabularnewline \hline \hline
			
			\multirow{2}{*}{\textit{HALF}} & 0,023 & -0,162 & 0,123 & -0,012 \tabularnewline
			& (0,204) & (0,250) & (0,211) & (0,255) \tabularnewline
			\hline
			
			\multirow{2}{*}{\textit{SCORE\_HALF}} & 0,175*** & 0,147*** & 0,175*** & 0,156*** \tabularnewline
			& (0,015) & (0,027) & (0,016) & (0,027) \tabularnewline
			\hline
			
			\multirow{2}{*}{\textit{SCORE\_HALF2}} & --- & 0,001 & --- & 0,002 \tabularnewline
			& & (0,001) & & (0,001) \tabularnewline
			\hline
			
			\multirow{2}{*}{\textit{SCORE\_HALF3}} & --- & 0,0001 & --- & 0,00009 \tabularnewline
			& & (0,0001) & & (0,0001) \tabularnewline
			\hline
			
			\multirow{2}{*}{\textit{BETTING\_RATE}} & --- & --- & -0,640*** & -0,646*** \tabularnewline
			& & & (0,086) & (0,087) \tabularnewline
			\hline
			
			$ Pseudo-R^{2} $ & 0,286 & 0,287 & 0,316 & 0,318 \tabularnewline
			\hline
			
			\textit{Pozorovania} & 1 650 & 1 650 & 1 650 & 1 650 \tabularnewline
			\hline
		\end{tabular}
		\egroup
		\caption{Vplyv prehrávania v polčase na výhru vo WNBA dátach}
		\vskip\abovecaptionskip\emph{Zdroj: <<vlastné spracovanie>>}
		\label{table:vplyvWNBA}	
	\end{table}				
	
	\begin{table}[H]
		\catcode`\-=12
		\centering 
		\footnotesize
		\bgroup
		\def\arraystretch{1.5}
		\begin{tabular}{|r||x{\sizeofcolumn}x{\sizeofcolumn}x{\sizeofcolumn}x{\sizeofcolumn}|}
			\hline
			
			\multirow{2}{*}{} & \multicolumn{4}{c|}{Závislá premenná: $ WIN_{i} $ = 1 ak domáci tím vyhral zápas} \tabularnewline \cline{2-5}
			& \textbf{(1)} & \textbf{(2)} & \textbf{(3)} & \textbf{(4)} \tabularnewline \hline \hline
			
			\multirow{2}{*}{\textit{HALF}} & 0,089 & -0,112 & 0,025 & -0,152 \tabularnewline
			& (0,164) & (0,201) & (0,173) & (0,212) \tabularnewline
			\hline
			
			\multirow{2}{*}{\textit{SCORE\_HALF}} & 0,186*** & 0,154*** & 0,166*** & 0,138*** \tabularnewline
			& (0,013) & (0,023) & (0,013) & (0,023) \tabularnewline
			\hline
			
			\multirow{2}{*}{\textit{SCORE\_HALF2}} & --- & -0,00002 & --- & 0,0001 \tabularnewline
			& & (0,0009) & & (0,0009) \tabularnewline
			\hline
			
			\multirow{2}{*}{\textit{SCORE\_HALF3}} & --- & 0,0002 & --- & 0,0001 \tabularnewline
			& & (0,0001) & & (0,0001) \tabularnewline
			\hline
			
			\multirow{2}{*}{\textit{BETTING\_RATE}} & --- & --- & -1,019*** & -1,014*** \tabularnewline
			& & & (0,084) & (0,085) \tabularnewline
			\hline
			
			$ Pseudo-R^{2} $ & 0,290 & 0,291 & 0,346 & 0,346 \tabularnewline
			\hline
			
			\textit{Pozorovania} & 2 627 & 2 627 & 2 627 & 2 627 \tabularnewline
			\hline
		\end{tabular}
		\egroup
		\caption{Vplyv prehrávania v polčase na výhru v EUROLIGA dátach}
		\vskip\abovecaptionskip\emph{Zdroj: <<vlastné spracovanie>>}
		\label{table:vplyvEURO}	
	\end{table}	
	
	Tvrdenie, že motivačný efekt bytia pozadu sa objavuje v dvoch z piatich líg, odvodené z grafického spracovania dát, sa potvrdilo aj vo výsledkoch spracovaných modelov viď tabuľky \ref{table:vplyvNBL} až \ref{table:vplyvEURO}. Štatistická nevýznamnosť premennej \textit{HALF} sa prejavila v súvislosti so všetkými NBL, WNBA a EUROLIGA. Nestalo sa, že by aspoň jeden z modelov vyšiel štatisticky významný. Motivačný efekt preto v týchto ligách nie je prítomný.
	
	Naopak, jednotlivé modely líg ŽBL a NBA vyšli všetky štatisticky významné, čo znamená, že v nich sa hľadaný motivačný efekt prejavil. Podkapitola \ref{sec:porovnanie} bližšie rozoberá spôsoby, akým sa tento efekt prejavil.

	\section{Porovnanie mužského a ženského basketbalu}
	\label{sec:porovnanie}
	Druhým cieľom tejto práce je porovnať mužský a ženský basketbal v súvislosti s existenciou a silou motivačného efektu bytia pozadu. Dataset obsahoval tri mužské ligy, NBL, NBA a EUROLIGU, a dve ženské ligy, ŽBL a WNBA. Keďže štatisticky významná nám vyšla jedna mužská a jedna ženská liga, porovnávané sú tieto dve – ŽBL a NBA.
	
	\begin{figure}[H]
		\centering
		\includegraphics[width=0.8\textwidth]{./grafy/ZBL4.png}
		\caption{Percento vyhraných zápasov domáceho tímu na základe bodového rozdielu pre dáta zo ŽBL}
		\vskip\abovecaptionskip\emph{Zdroj: <<vlastné spracovanie>>}
		\label{fig:ZBL4}
	\end{figure}
	
	Ženská liga ŽBL mala spomedzi všetkých líg najmenšiu vzorku pozorovaní, zhruba dvadsaťsedem krát menšiu ako NBA liga. Z grafu \ref{fig:ZBL4} je vidieť, že čím viac tímy ŽBL vyhrávajú, tým je vyššia pravdepodobnosť výhry v celom zápase. Napríklad tímy, ktoré vedú v polčase o 6 bodov vyhrávajú viac ako 80 \% času.
	
	\begin{table}[h]
		\catcode`\-=12
		\centering 
		\footnotesize
		\bgroup
		\def\arraystretch{1.5}
		\begin{tabular}{|r||x{\sizeofcolumn}x{\sizeofcolumn}x{\sizeofcolumn}x{\sizeofcolumn}|}
			\hline
			
			\multirow{2}{*}{} & \multicolumn{4}{c|}{Závislá premenná: $ WIN_{i} $ = 1 ak domáci tím vyhral zápas} \tabularnewline \cline{2-5}
			& \textbf{(1)} & \textbf{(2)} & \textbf{(3)} & \textbf{(4)} \tabularnewline \hline \hline
			
			\multirow{2}{*}{\textit{HALF}} & -0,801** & -0,917** & -0,754* & -0,799* \tabularnewline
			& (0,393) & (0,462) & (0,422) & (0,484) \tabularnewline
			\hline
			
			\multirow{2}{*}{\textit{SCORE\_HALF}} & 0,182*** & 0,163*** & 0,166*** & 0,159*** \tabularnewline
			& (0,028) & (0,048) & (0,031) & (0,050) \tabularnewline
			\hline
			
			\multirow{2}{*}{\textit{SCORE\_HALF2}} & --- & -0,0003* & --- & 0,001 \tabularnewline
			& & (0,002) & & (0,002) \tabularnewline
			\hline
			
			\multirow{2}{*}{\textit{SCORE\_HALF3}} & --- & 0,00008 & --- & 0,00002 \tabularnewline
			& & (0,0002) & & (0,0002) \tabularnewline
			\hline
			
			\multirow{2}{*}{\textit{BETTING\_RATE}} & --- & --- & -0,364*** & -0,369*** \tabularnewline
			& & & (0,079) & (0,080) \tabularnewline
			\hline
			
			$ Pseudo-R^{2} $ & 0,484 & 0,484 & 0,530 & 0,530 \tabularnewline
			\hline
			
			\textit{Pozorovania} & 551 & 551 & 551 & 551 \tabularnewline
			\hline
		\end{tabular}
		\egroup
		\caption{Vplyv prehrávania v polčase na výhru v ŽBL dátach}
		\vskip\abovecaptionskip\emph{Zdroj: <<vlastné spracovanie>>}
		\label{table:vplyvZBL}
	\end{table}	

	Štatisticky významné vyšli všetky štyri spracované modely čo znamená, že motivačný efekt je v týchto dátach prítomný. Túto skutočnosť potvrdzuje aj pohľad na graf \ref{fig:ZBL4}, kde je vidieť silná diskontinuita v okolí nuly. Znamienko premennej \textit{HALF} v tabuľke \ref{table:vplyvZBL} je však opačné ako v štúdií Bergera a Popa (\citeyear{berger2011}) – z toho sa dá vyvodiť, že akonáhle je tím pozadu v polčase, pravdepodobnosť výhry celého zápasu je nižšia. V prípade, že tím prehráva o jeden bod, vyhrá zápas zhruba 40 \% času, zatiaľ čo ak tím z tejto ligy vyhráva o jeden bod, pravdepodobnosť výhry sa zvyšuje až na približne 67 \%.

	Inými slovami sa dá povedať, že tím, ktorý prehráva v polčase o jeden bod vyhrá zápas menej pravdepodobne ako jeho súper. Tímom, ktoré prehrávajú v polčase, sa po prepočte koeficientov premennej \textit{HALF} na hodnoty medzného efektu, znižuje pravdepodobnosť výhry o 7,6 až 10,3 \%, viď tabuľka \ref{table:prehladvplyvu}. To znamená, že byť pozadu v ŽBL nijako nezvyšuje motiváciu ani šancu na výhru, práve naopak.
	
	\begin{table}[H]
		\catcode`\-=12
		\centering 
		\bgroup
		\def\arraystretch{1.5}
		\begin{tabular}{|l||x{\sizeofcolumn}x{\sizeofcolumn}x{\sizeofcolumn}x{\sizeofcolumn}|}
			\hline
			& \textbf{(1)} & \textbf{(2)} & \textbf{(3)} & \textbf{(4)} \tabularnewline \hline \hline
			
			\textit{HALF} pre ŽBL & -0,09051 & -0,1039 & -0,0767 & -0,08148 \tabularnewline
			\hline 
			
			\textit{HALF} pre NBA & 0,0581  & 0,0471 & 0,05436 & 0,04649 \tabularnewline
			\hline
		\end{tabular}
		\egroup
		\caption{Percentuálny vplyv premennej \textit{HALF} pre ŽBL a NBA modely}
		\vskip\abovecaptionskip\emph{Zdroj: <<vlastné spracovanie>>}
		\label{table:prehladvplyvu}
	\end{table}	
	
	Mužskou ligou, ktorú porovnávame so ŽBL, je americká NBA, ktorej dataset bol najväčší spomedzi všetkých piatich líg. Tvoril až 14 970 zápasov. Rovnako ako ŽBL, aj modely NBA vyšli všetky štatisticky významné. Platí aj priama úmera medzi výškou vedenia v polčase a pravdepodobnosťou výhry zápasu. Z grafu \ref{fig:NBA4} vidno, že tímy, ktoré vedú v polčase o šesť bodov vyhrávajú v NBA 75 \% prípadov, čo je menej ako v ŽBL.
	
	\begin{figure}[H]
		\centering
		\includegraphics[width=0.8\textwidth]{./grafy/NBA4.png}
		\caption{Percento vyhraných zápasov domáceho tímu na základe bodového rozdielu pre dáta z NBA}
		\vskip\abovecaptionskip\emph{Zdroj: <<vlastné spracovanie>>}
		\label{fig:NBA4}
	\end{figure}
	
	V tabuľke \ref{table:vplyvNBAja} má premenná \textit{HALF} kladné znamienko – akonáhle je tím pozadu v polčase, tak je pravdepodobnosť výhry v zápase vyššia. Skok v okolí nuly v grafe \ref{fig:NBA4} nie je natoľko výrazný ako v prípade ŽBL, ale existuje. Keď tím prehráva o jeden bod, vyhrá zápas 58 \% prípadov, zatiaľ čo ak tím vyhráva o jeden bod, zápas vyhrá 53 \% prípadov.	
		
	\begin{table}[H]
		\catcode`\-=12
		\centering
		\footnotesize
		\bgroup
		\def\arraystretch{1.5}
		\begin{tabular}{|r||x{\sizeofcolumn}x{\sizeofcolumn}x{\sizeofcolumn}x{\sizeofcolumn}|}
			\hline
			
			\multirow{2}{*}{} & \multicolumn{4}{c|}{Závislá premenná: $ WIN_{i} $ = 1 ak domáci tím vyhral zápas} \tabularnewline \cline{2-5}
			& \textbf{(1)} & \textbf{(2)} & \textbf{(3)} & \textbf{(4)} \tabularnewline \hline \hline
			
			\multirow{2}{*}{\textit{HALF}} & 0,345*** & 0,279*** & 0,352*** & 0,301*** \tabularnewline
			& (0,066) & (0,079) & (0,069) & (0,082) \tabularnewline
			\hline
			
			\multirow{2}{*}{\textit{SCORE\_HALF}} & 0,170*** & 0,160*** & 0,160*** & 0,153*** \tabularnewline
			& (0,005) & (0,008) & (0,005) & (0,008) \tabularnewline
			\hline
			
			\multirow{2}{*}{\textit{SCORE\_HALF2}} & --- & -0,00004 & --- & 0,00001 \tabularnewline
			& & (0,0003) & & (0,0003) \tabularnewline
			\hline
			
			\multirow{2}{*}{\textit{SCORE\_HALF3}} & --- & 0,00004 & --- & 0,00003 \tabularnewline
			& & (0,00003) & & (0,00003) \tabularnewline
			\hline
			
			\multirow{2}{*}{\textit{BETTING\_RATE}} & --- & --- & -1,118*** & -1,117*** \tabularnewline
			& & & (0,038) & (0,038) \tabularnewline
			\hline
			
			$ Pseudo-R^{2} $ & 0,247 & 0,247 & 0,299 & 0,299 \tabularnewline
			\hline
			
			\textit{Pozorovania} & 14 970 & 14 970 & 14 970 & 14 970 \tabularnewline
			\hline
		\end{tabular}
		\egroup
		\caption{Vplyv prehrávania v polčase na výhru v NBA dátach}
		\vskip\abovecaptionskip\emph{Zdroj: <<vlastné spracovanie>>}
		\label{table:vplyvNBAja}	
	\end{table}	

	V tomto datasete sa potvrdilo, že byť mierne pozadu, napríklad o jeden alebo dva body, zvyšuje motiváciu a tým aj šancu na výhru. V percentuálnom zobrazení sa tímom, ktoré prehrávajú v polčase zvyšuje pravdepodobnosť výhry o 4,6 až 5,8 percent, viď tabuľka \ref{table:prehladvplyvu}. Tieto hodnoty boli získané po prepočte koeficientov premennej \textit{HALF} na hodnoty medzného vplyvu.
	
	\section{Porovnanie favoritov a tých ostatných}
	Tretím cieľom je porovnať výsledky favoritov a tých ostatných a zistiť prípadný rozdiel medzi týmito dvomi skupinami. Na základe predchádzajúceho testovania vyšla ako jediná štatisticky významná a zároveň obsahujúca pozitívny motivačný efekt liga NBA, ktorá je z tohto dôvodu zvolená pre toto testovanie. V dátach NBA sa prejavil motivačný efekt bytia pozadu, na grafe \ref{fig:NBA4} ilustrovaný skokom lineárnej priamky dát v bode nulového rozdielu medzi domácim a hosťovským tímom v polčase. Táto časť práce sa snaží zistiť, či je tento skok ovplyvnený veľkosťou kurzu a teda tým, akým veľkým favoritom je domáci tím. Čím nižší je kurz na domáci tím, tým väčším je favoritom a vice versa. 

	Zvažované boli dve formy testovania. Prvou bolo rozdelenie datasetu na dve skupiny, favoritov a tých ostatných, prostredníctvom tzv. cut offu v dátach, v premennej \textit{BETTING\_RATE}. Cut offom sa rozumie akýsi rez v dátach, ktorý ich rozdelí na dve časti. V tomto prípade by bol cut off urobený tak, aby obe skupiny obsahovali rozumný počet pozorovaní. Napríklad ak by skupinu favoritov malo tvoriť približne 30 \% najväčších domácich favoritov, cut off by bol urobený na kurze 0,36. Všetky tímy s kurzom menším ako 0,36 by boli v skupine favoritov a všetky tímy s kurzom väčším ako 0,36 by boli v skupine ostatní. V tomto prípade by však bola otázna vhodná veľkosť obidvoch skupín.

	Druhou formou, ktorá bola aj zvolená, bolo použitie celého datasetu bez jeho delenia na skupiny a bez potreby robiť cut off na dátach. Riešením je pridaním interakčnej premennej \textit{BETTING\_RATE * HALF}.  Prostredníctvom interakčnej premennej bude možné zistiť, ako sa odlišuje motivačný efekt, keď je niekto väčší alebo menší favorit. Inými slovami, koeficient interakčnej premennej hovorí, ako sa efekt odchyľuje pre niekoho, kto má nadpriemerný alebo podpriemerný kurz. 
	
	Postup bol nasledovný: do základného ekonometrického modelu z podkapitoly \ref{sec:pouzitymodel} bola pridaná nová vysvetľujúca premenná \textit{BETTING\_RATE * HALF}, ktorá vznikla ako súčin nanormovanej premennej \textit{BETTING\_RATE}~\footnote{Kurz mínus jeho priemer a to celé vydelené smerodatnou odchýlkou kurzu.} a premennej \textit{HALF}. Nová podoba modelu využitého v tejto časti je:
	
	\begin{multline}
	WIN_{i} = \alpha + \beta _{1} [\textit{HALF}]_{i} + \beta _{2} [\textit{SCORE~\_~HALF}]_{i} + \beta _{3} [\textit{SCORE~\_~HALF2}]_{i} \\
	+ \beta _{4} [\textit{SCORE~\_~HALF3}]_{i} + \beta _{5} [\textit{BETTING\_RATE}]_{i} + \beta _{6} [\textit{BETTING\_RATE * HALF}]_{i} + \varepsilon_{i}.
	\end{multline}
	
	Očakávanie je, že motivačný efekt v prípade, že je tím pozadu v polčase je väčší a silnejší pre väčšieho favorita a teda pre tím s nižším kurzom. Ak teda vyjde koeficient interakčnej premennej \textit{BETTING\_RATE * HALF} štatisticky významný a záporný, očakávanie sa potvrdí. 
	
	Pomyselné grafy vhodné na interpretáciu kladného a záporného koeficientu interakčnej premennej pozostávajú z osi x, na ktorej sa nachádza kurz domáceho tímu. Čím ju kurz nižší, tým ide o väčšieho favorita a čím je vyšší, tým je väčší outsider. Na osi y je koeficient premennej \textit{HALF}, čiže veľkosť motivačného efektu. Keď vyjde koeficient kladný, priamka v grafe je rastúca. To znamená, že domáce tímy s väčším kurzom majú väčší koeficient. V tomto prípade je motivačný efekt väčší u outsiderov. Ak však vyjde koeficient záporný, znamená to, že motivačný efekt bytia pozadu v polčase je väčší u favoritov.
	
	V prípade, že premenná \textit{BETTING\_RATE * HALF} vyjde štatisticky nevýznamná, efekt nebude závisieť na veľkosti kurzu a tom, akým veľkým favoritom je tím.
	
	\newcommand{\twosizeofcolumn}{12em}
	\begin{table}[h]
		\catcode`\-=12
		\centering
		\footnotesize 
		\bgroup
		\def\arraystretch{1.5}
		\begin{tabular}{|r||x{\twosizeofcolumn}x{\twosizeofcolumn}|}
			\hline
			
			\multirow{2}{*}{} & \multicolumn{2}{c|}{Závislá premenná: $ WIN_{i} $ = 1 ak domáci tím vyhral zápas} \tabularnewline \cline{2-3}
			& \textbf{(5)} & \textbf{(6)} \tabularnewline \hline \hline
			
			\multirow{2}{*}{\textit{HALF}} & 0,353*** & 0,302*** \tabularnewline
			& (0,069) & (0,082) \tabularnewline
			\hline
			
			\multirow{2}{*}{\textit{SCORE\_HALF}} & 0,161*** & 0,153*** \tabularnewline
			& (0,005) & (0,008) \tabularnewline
			\hline
			
			\multirow{2}{*}{\textit{SCORE\_HALF2}} & --- & 0,00002 \tabularnewline
			& & (0,0003) \tabularnewline
			\hline
			
			\multirow{2}{*}{\textit{SCORE\_HALF3}} & --- & 0,00003 \tabularnewline
			& & (0,00003) \tabularnewline
			\hline
			
			\multirow{2}{*}{\textit{BETTING\_RATE}} & -1,164*** & -1,163*** \tabularnewline
			& (0,050) & (0,050) \tabularnewline
			\hline
			
			\multirow{2}{*}{\textit{BETTING\_RATE * HALF}} & 0,072 & 0,073 \tabularnewline
			& (0,052) & (0,052) \tabularnewline
			\hline
			
			$ Pseudo-R^{2} $ & 0,299 & 0,299 \tabularnewline
			\hline
			
			\textit{Pozorovania} & 14 970 & 14 970 \tabularnewline
			\hline
		\end{tabular}
		\egroup
		\caption{Vplyv prehrávania v polčase na výhru – favoriti VS ostatní}
		\vskip\abovecaptionskip\emph{Zdroj: <<vlastné spracovanie>>}
		\label{table:vplyvfvo}	
	\end{table}	
	
	Spracované boli iba dva modely a to (5) a (6), čo sú v podstate modely (3) a (4) z predchádzajúcich podkapitol \ref{sec:overenie} a \ref{sec:porovnanie} s pridaním interakčnej premennej \textit{BETTING\_RATE~ *~HALF}.  Modely (1) a (2) neboli použité, keďže neobsahujú premennú kurzu, o ktorú sa interakčná premenná kontroluje.
	
	Výstupom tabuľky \ref{table:vplyvfvo} a grafu \ref{fig:NBA_fvo} je, že motivačný efekt bytia pozadu v polčase sa nemení na základe kurzu a teda toho, akým favoritom je tím. Koeficient interakčnej premennej v modeli (3) a (4) vyšiel s malou kladnou hodnotou, ktorá však nie je štatisticky významná. Veľkosť favorita tým pádom nemá vplyv na veľkosť motivačného efektu a očakávanie sa nepotvrdilo.
	
	\begin{figure}[h]
		\centering
		\includegraphics[width=0.8\textwidth]{./grafy/NBA_fvo.png}
		\caption{Percento vyhraných zápasov domáceho tímu na základe bodového rozdielu pre dáta z NBA – favoriti VS ostatní}
		\vskip\abovecaptionskip\emph{Zdroj: <<vlastné spracovanie>>}
		\label{fig:NBA_fvo}
	\end{figure}
	
	\clearpage
	\chapter{Záver}
	Cieľom tejto bakalárskej práce bolo v prvom rade overiť robustnosť výsledkov štúdie Bergera a Popa (2011) o tom, že prehrávanie zápasu v polčase má dostatočne silný motivačný účinok na zvrátenie priebehu zápasu pre tím, ktorý je pozadu.
	
	Druhým cieľom bolo preskúmanie rozdielnosti pôsobenia motivačného efektu v mužskom a ženskom basketbale, tzn. že snahou tejto práce bolo zistiť, či efekt existuje v obidvoch prípadoch a či je jeho sila porovnateľná alebo rozdielna.
	
	Tretím cieľom bolo zohľadniť rolu favorita a zistiť, či sa líši motivačný efekt favoritov od motivačného efektu tých ostatných.
	
	Analýza dát, ktoré boli k dispozícií, bola uskutočnená prostredníctvom modelu regresnej diskontinuity a logit modelu v programe \textit{R Studio}.
	
	Na základe prieskumu dát z piatich basketbalových líg vzišiel záver, že motivačný efekt priebežného výsledku nie je robustný. Rovnaký motivačný efekt ako zistili Berger a Pope sa ukázal iba v prípade jednej lige. Ide pritom len o potvrdenie tej istej ligy – NBA. Výsledky pre ligu NBA potvrdili, že byť trochu pozadu v polčase zvyšuje motiváciu a tým aj šancu na výhru v zápase v priemere až o 5,15 \%. V prípade, že tím prehráva o jeden bod, vyhrá zápas v 58 \% prípadov, zatiaľ čo ak o jeden bod vyhráva, vyhrá celý zápas v  približne 53 \% prípadov.
	
	Pri analýze dát v súvislosti s rozdielom medzi pohlaviami bol zistený rozdiel. Zastúpením mužského basketbalu bola NBA, v ktorej, motivačný efekt existuje. Reprezentantom ženského basketbalu bola ŽBL, v ktorej sa síce motivačný efekt prejavil, avšak mal opačný smer pôsobenia. Namiesto toho, aby zvýhodňoval tímy tesne prehrávajúce, ako tomu je v prípade mužov, väčšiu pravdepodobnosť výhry majú tímy tesne vyhrávajúce.
	
	Okrem toho bolo zistené, že očakávanie z úvodu práce, týkajúce sa favoritov a tých ostatných, nebolo správne. Dáta nepreukázali, že by motivačný efekt pôsobil silnejším a výraznejším spôsobom na tie tímy, ktoré by boli v zápase favorizované.
	
	Možno teda konštatovať, že ak sa motivačný efekt prejaví, tak skôr u mužov ako u žien. Ak sa však vyskytne v ženskom basketbale, má opačný efekt. Rozdiel v pôsobení efektu medzi favoritmi a tými ostatnými nebol preukázaný.
	


\printbibliography[heading=bibintoc] %% Print the bibliography.

\listoffigures
\addcontentsline{toc}{chapter}{Zoznam grafov}	


\listoftables
\addcontentsline{toc}{chapter}{Zoznam tabuliek}


	\chapter*{Zoznam skratiek}
	\addcontentsline{toc}{chapter}{Zoznam skratiek}
	{\renewcommand\labelitemi{}
	\begin{itemize}
	\item \textbf{EURO} ~~~ Euroliga
	\item \textbf{NBA} ~~~ National Basketball Association
	\item \textbf{NBL} ~~~ Národná basketbalová liga
	\item \textbf{RDD} ~~~ Regression Discontinuity Design
	\item \textbf{WNBA} ~~~ Women's National Basketball Association
	\item \textbf{ŽBL} ~~~ Ženská basketbalová liga
	\end{itemize}}

\appendix{} %% Start the appendices.
\counterwithin{figure}{chapter}
\counterwithin{table}{chapter}

\chapter{Príloha grafov}
	
	\renewcommand{\figurename}{Graf}
	\begin{figure}[!ht]
		\centering
		\captionof{figure}{Percento vyhraných zápasov domáceho tímu na základe bodového rozdielu pre dáta z NCAA – štúdia Berger a Pope}
		\includegraphics[width=0.8\textwidth]{./grafy/NCAA.png}
		\vskip\abovecaptionskip\emph{Zdroj: <<\cite{berger2011} a vlastné spracovanie>>}
		\label{fig:NCAA1}
	\end{figure}


	\begin{figure}[!ht]
		\centering
		\includegraphics[width=0.8\textwidth]{./grafy/NBL4.png}
		\caption{Percento vyhraných zápasov domáceho tímu na základe bodového rozdielu pre dáta z NBL}
		\vskip\abovecaptionskip\emph{Zdroj: <<vlastné spracovanie>>}
		\label{fig:NBL4}
	\end{figure}

	\begin{figure}[!ht]
		\centering
		\includegraphics[width=0.8\textwidth]{./grafy/WNBA4.png}
		\caption{Percento vyhraných zápasov domáceho tímu na základe bodového rozdielu pre dáta z WNBA}
		\vskip\abovecaptionskip\emph{Zdroj: <<vlastné spracovanie>>}
		\label{fig:WNBA4}
	\end{figure}
	
	\begin{figure}[!ht]
		\centering
		\includegraphics[width=0.8\textwidth]{./grafy/EURO4.png}
		\caption{Percento vyhraných zápasov domáceho tímu na základe bodového rozdielu pre dáta z EUROLIGY}
		\vskip\abovecaptionskip\emph{Zdroj: <<vlastné spracovanie>>}
		\label{fig:EURO4}
	\end{figure}

\clearpage

\chapter{Príloha tabuliek}

	\newcolumntype{x}[1]{>{\centering\hspace{0pt}}p{#1}}
	\begin{table}[!ht]
		\catcode`\-=12
		\centering
		\footnotesize 
		\bgroup
		\def\arraystretch{1.5}
		\begin{tabular}{|l||x{\sizeofcolumn}x{\sizeofcolumn}x{\sizeofcolumn}x{\sizeofcolumn}|}
			\hline
			
			\multirow{2}{*}{} & \multicolumn{4}{c|}{Závislá premenná: $ win_{i} $ = 1 ak domáci tím vyhral zápas} \tabularnewline \cline{2-5}
			& (1) & (2) & (3) & (4) \tabularnewline \hline \hline
			
			\multirow{2}{*}{\textit{Prehrávanie v polčase}} & 0,025*** & 0,023* & 0,025*** & 0,021 \tabularnewline
			& (0,010) & (0,014) & (0,009) & (0,013) \tabularnewline
			\hline
			
			\textit{Domáci tím} &  &  & 0,0057*** & 0,0057*** \tabularnewline
			\multicolumn{1}{|r||}{\textit{výherné percento}} & & & (0,0001) & (0,001) \tabularnewline
			\hline 
			
			\textit{Hosťujúci tím} &  &  & -0,0055*** & -0,0055*** \tabularnewline
			\multicolumn{1}{|r||}{\textit{výherné percento}} & & & (0,0001) & (0,0001) \tabularnewline
			\hline
			
			\textit{Bodový rozdiel v polčase} & X & X & X & X \tabularnewline
			\multicolumn{1}{|r||}{(\textit{lineárny})} & & & & \tabularnewline
			\hline
			
			\textit{Bodový rozdiel v polčase} & & X & & X \tabularnewline
			\multicolumn{1}{|r||}{(\textit{kubický})} & & & & \tabularnewline
			\hline
			
			$ Pseudo-R^{2} $ & 0,143 & 0,144 & 0,207 & 0,208 \tabularnewline
			\hline
			
			\textit{Pozorovania} & 29 159 & 29 159 & 28 808 & 28 808 \tabularnewline
			\hline
		\end{tabular}
		\egroup
		\caption{Vplyv prehrávania v polčase na výhru v NCAA dátach}
		\vskip\abovecaptionskip\emph{Zdroj: <<\cite{berger2011} a vlastné spracovanie>>}
		\vskip\abovecaptionskip\emph{Pozn.: V tabuľke sú uvedené hodnoty medzného efektu koeficientov jednotlivých premenných a v zátvorkách je ich smerodatná chyba z použitia logit modelu.}
		\label{table:vplyvNCAA}
	\end{table}	

%%\thispagestyle{empty}
%%\clearpage
%%\pagenumbering{arabic}

\end{document}